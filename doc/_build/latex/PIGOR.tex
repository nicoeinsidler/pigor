%% Generated by Sphinx.
\def\sphinxdocclass{report}
\documentclass[letterpaper,10pt,english]{sphinxmanual}
\ifdefined\pdfpxdimen
   \let\sphinxpxdimen\pdfpxdimen\else\newdimen\sphinxpxdimen
\fi \sphinxpxdimen=.75bp\relax

\PassOptionsToPackage{warn}{textcomp}
\usepackage[utf8]{inputenc}
\ifdefined\DeclareUnicodeCharacter
% support both utf8 and utf8x syntaxes
\edef\sphinxdqmaybe{\ifdefined\DeclareUnicodeCharacterAsOptional\string"\fi}
  \DeclareUnicodeCharacter{\sphinxdqmaybe00A0}{\nobreakspace}
  \DeclareUnicodeCharacter{\sphinxdqmaybe2500}{\sphinxunichar{2500}}
  \DeclareUnicodeCharacter{\sphinxdqmaybe2502}{\sphinxunichar{2502}}
  \DeclareUnicodeCharacter{\sphinxdqmaybe2514}{\sphinxunichar{2514}}
  \DeclareUnicodeCharacter{\sphinxdqmaybe251C}{\sphinxunichar{251C}}
  \DeclareUnicodeCharacter{\sphinxdqmaybe2572}{\textbackslash}
\fi
\usepackage{cmap}
\usepackage[T1]{fontenc}
\usepackage{amsmath,amssymb,amstext}
\usepackage{babel}
\usepackage{times}
\usepackage[Bjarne]{fncychap}
\usepackage{sphinx}

\fvset{fontsize=\small}
\usepackage{geometry}

% Include hyperref last.
\usepackage{hyperref}
% Fix anchor placement for figures with captions.
\usepackage{hypcap}% it must be loaded after hyperref.
% Set up styles of URL: it should be placed after hyperref.
\urlstyle{same}
\addto\captionsenglish{\renewcommand{\contentsname}{Contents:}}

\addto\captionsenglish{\renewcommand{\figurename}{Fig.\@ }}
\makeatletter
\def\fnum@figure{\figurename\thefigure{}}
\makeatother
\addto\captionsenglish{\renewcommand{\tablename}{Table }}
\makeatletter
\def\fnum@table{\tablename\thetable{}}
\makeatother
\addto\captionsenglish{\renewcommand{\literalblockname}{Listing}}

\addto\captionsenglish{\renewcommand{\literalblockcontinuedname}{continued from previous page}}
\addto\captionsenglish{\renewcommand{\literalblockcontinuesname}{continues on next page}}
\addto\captionsenglish{\renewcommand{\sphinxnonalphabeticalgroupname}{Non-alphabetical}}
\addto\captionsenglish{\renewcommand{\sphinxsymbolsname}{Symbols}}
\addto\captionsenglish{\renewcommand{\sphinxnumbersname}{Numbers}}

\addto\extrasenglish{\def\pageautorefname{page}}

\setcounter{tocdepth}{1}



\title{PIGOR Documentation}
\date{Oct 17, 2019}
\release{}
\author{Nico Einsidler}
\newcommand{\sphinxlogo}{\vbox{}}
\renewcommand{\releasename}{}
\makeindex
\begin{document}

\pagestyle{empty}
\sphinxmaketitle
\pagestyle{plain}
\sphinxtableofcontents
\pagestyle{normal}
\phantomsection\label{\detokenize{index::doc}}


Pigor is a lightweight analysis tool for the the polarimeter instrument NEPTUN beam port of the 250kW research reactor hosted at \sphinxhref{https://ati.tuwien.ac.at/startseite/}{Atominstitut} of \sphinxhref{https://www.tuwien.ac.at/}{TU Wien}, Austria. For more information visit \sphinxhref{http://www.neutroninterferometry.com/}{our homepage}.


\chapter{How to install PIGOR}
\label{\detokenize{installation:how-to-install-pigor}}\label{\detokenize{installation::doc}}

\section{Installing PIGOR}
\label{\detokenize{installation:installing-pigor}}\label{\detokenize{installation:installation}}
Assuming you have a running installation of \sphinxstylestrong{Python 3.7} or higher, to install PIGOR please follow these steps:
\begin{enumerate}
\def\theenumi{\arabic{enumi}}
\def\labelenumi{\theenumi .}
\makeatletter\def\p@enumii{\p@enumi \theenumi .}\makeatother
\item {} \begin{description}
\item[{Download the source from Bitbucket with \sphinxcode{\sphinxupquote{git clone https://github.com/nicoeinsidler/pigor.git}}. This should include (at least) the following files:}] \leavevmode\begin{itemize}
\item {} 
\sphinxcode{\sphinxupquote{measurement.py}}

\item {} 
\sphinxcode{\sphinxupquote{pigor.py}}

\item {} 
\sphinxcode{\sphinxupquote{README.md}}

\item {} 
\sphinxcode{\sphinxupquote{requirements.py}}

\item {} 
\sphinxcode{\sphinxupquote{requirements-dev.py}}

\item {} 
\sphinxcode{\sphinxupquote{requirements-measurement.py}}

\end{itemize}

\end{description}

\item {} 
Install the requirements with:

\begin{sphinxVerbatim}[commandchars=\\\{\}]
\PYG{n}{pip} \PYG{n}{install} \PYG{o}{\PYGZhy{}}\PYG{n}{r} \PYG{n}{requirements}\PYG{o}{.}\PYG{n}{txt}
\end{sphinxVerbatim}

\end{enumerate}

\begin{sphinxadmonition}{note}{Note:}
If you are using two separate python 2.x and 3.x installations, you might need to use \sphinxcode{\sphinxupquote{pip3}} instead of \sphinxcode{\sphinxupquote{pip}}.
\end{sphinxadmonition}

You can now run PIGOR with the following command in the folder where the \sphinxcode{\sphinxupquote{pigor.py}} and \sphinxcode{\sphinxupquote{measurement.py}} files are located:

\begin{sphinxVerbatim}[commandchars=\\\{\}]
\PYG{n}{python} \PYG{n}{pigor}\PYG{o}{.}\PYG{n}{py}
\end{sphinxVerbatim}

\begin{sphinxadmonition}{note}{Note:}
If you only want to use the {\hyperref[\detokenize{measurement:measurement.Measurement}]{\sphinxcrossref{\sphinxcode{\sphinxupquote{measurement.Measurement}}}}} class, you can only install it’s dependencies with \sphinxcode{\sphinxupquote{pip install -r requirements-measurement.txt}}. But since most dependencies are shared, it won’t make much of a difference.
\end{sphinxadmonition}


\section{Installing only the Measurement Class}
\label{\detokenize{installation:installing-only-the-measurement-class}}
\begin{sphinxadmonition}{warning}{Warning:}
This is not maintained yet. Please use this feature with caution and build it first.
\end{sphinxadmonition}

In some cases the user may only want to install the measurement module to use the Measurement class as one of their python installations modules. A zip or tarball file should be provided for this use case. See \DUrole{xref,std,std-ref}{deployment-measurement-module}.

To install (and copy into the user’s installed modules) simply unpack the zip or tarball. Head over to its directory and type from there:

\begin{sphinxVerbatim}[commandchars=\\\{\}]
python setup.py install
\end{sphinxVerbatim}


\section{Installation for Developers}
\label{\detokenize{installation:installation-for-developers}}
A known working set of python modules used to create PIGOR can be found in \sphinxcode{\sphinxupquote{requirements-dev.txt}}. To install all these requirements, simply type:

\begin{sphinxVerbatim}[commandchars=\\\{\}]
\PYG{n}{pip} \PYG{n}{install} \PYG{o}{\PYGZhy{}}\PYG{n}{r} \PYG{n}{requirements}\PYG{o}{\PYGZhy{}}\PYG{n}{dev}\PYG{o}{.}\PYG{n}{txt}
\end{sphinxVerbatim}

It is recommended to use a virtual environment.


\chapter{Measurement Naming Convention}
\label{\detokenize{naming-convention:measurement-naming-convention}}\label{\detokenize{naming-convention::doc}}
In order for the Measurement class to work properly, the user should follow this naming convention:

\sphinxcode{\sphinxupquote{YYYY\_MM\_DD\_HHmm\_IDENTIFIER\_TYPE.dat}} (not case sensitive)

where…
\begin{itemize}
\item {} \begin{description}
\item[{\sphinxcode{\sphinxupquote{IDENTIFIER}} denotes the type of measurement or the reason for the measurement. The following values are common and recognized by PIGOR:}] \leavevmode\begin{itemize}
\item {} 
\sphinxcode{\sphinxupquote{dcX\#}}: DC coil (for the magnetic field in x-direction) scan for the DC coil with the number \#. A sine fit is automatically applied, when \sphinxcode{\sphinxupquote{TYPE}} is not set.

\item {} 
\sphinxcode{\sphinxupquote{dcZ\#}}: DC coil scan in z-direction (compensation field) for DC coil number \#. Fitted with a polynomial of order 5 by default.

\item {} 
\sphinxcode{\sphinxupquote{pos\#}}: position scan with stage number \#. Fitted with a Gaussian by default.

\end{itemize}

\end{description}

\item {} \begin{description}
\item[{\sphinxcode{\sphinxupquote{TYPE}} specifies and overrides the type of fitting/analysis that should be applied by PIGOR. Possible values:}] \leavevmode\begin{itemize}
\item {} \begin{description}
\item[{fitting types}] \leavevmode\begin{itemize}
\item {} 
\sphinxcode{\sphinxupquote{poly5}}: polynom of order 5

\item {} 
\sphinxcode{\sphinxupquote{sine\_lin}}: sine plus additional added linear function

\item {} 
\sphinxcode{\sphinxupquote{sine}}: sine

\item {} 
\sphinxcode{\sphinxupquote{gauss}}: Gauß fit

\end{itemize}

\end{description}

\item {} \begin{description}
\item[{special analysis types}] \leavevmode\begin{itemize}
\item {} 
\sphinxcode{\sphinxupquote{pol}}: specifies that this measurement is a degree of polarisation measurement. This will automatically calculate the degree of polarisation for any given number of points (number of data points must be multiple of four)

\end{itemize}

\end{description}

\end{itemize}

\end{description}

\end{itemize}

Types can be combined within a types group (gauss and pol for example, like “gauss\_pol” or “pol\_gauss”).


\chapter{Quickstart Guide}
\label{\detokenize{quickstart:quickstart-guide}}\label{\detokenize{quickstart::doc}}
\begin{sphinxadmonition}{note}{Note:}
Please refer to the {\hyperref[\detokenize{installation:installation}]{\sphinxcrossref{\DUrole{std,std-ref}{Installing PIGOR}}}} page for more information on the installation.
\end{sphinxadmonition}

Suppose you have some measurement data files with in a directory. The hierachy looks like this:
\begin{itemize}
\item {} \begin{description}
\item[{measurements}] \leavevmode\begin{itemize}
\item {} \begin{description}
\item[{2019-01}] \leavevmode\begin{itemize}
\item {} 
2019-01-01-dc1x.dat

\item {} 
2019-01-04-dc1z.dat

\item {} 
2019-01-05-dc1z.dat

\end{itemize}

\end{description}

\item {} \begin{description}
\item[{2019-02}] \leavevmode\begin{itemize}
\item {} 
2019-02-10-dc2x.dat

\item {} 
2019-02-11-dc2x.dat

\item {} 
2019-02-14-dc2z.dat

\item {} 
2019-01-19-dc2z.dat

\item {} 
2019-01-20-dc2z.dat

\item {} 
2019-01-28-dc1x.dat

\end{itemize}

\end{description}

\item {} \begin{description}
\item[{2019-03}] \leavevmode\begin{itemize}
\item {} 
2019-02-02-dc3x.dat

\item {} 
2019-02-03-dc3z.dat

\end{itemize}

\end{description}

\end{itemize}

\end{description}

\end{itemize}

\begin{sphinxadmonition}{note}{Note:}
Here we will assume that \sphinxcode{\sphinxupquote{python}} will start Python 3. On some installations it must be explicitly started with \sphinxcode{\sphinxupquote{python3}}.
\end{sphinxadmonition}

In order to analyse these files with PIGOR, head to the directory where PIGOR is located. Start it by executing the command \sphinxcode{\sphinxupquote{python pigor.py}}. If it is the first startup it will go through some questions regarding your analysis setup. The most important question is the first one: Where should PIGOR look for files to analyse? This is the folder where all your measurements are located in and will be refered as PIGOR’s root folder. In our example type in the path to \sphinxcode{\sphinxupquote{measurements/}}. You can either use absolute paths or relative ones. PIGOR will automatically check if the path actually exists, if not you will be prompted to enter a new one again. It also does not matter if you are using a POSIX type (IEEE Std 1003.1-1988) or as a DOS type path. (POSIX uses / and DOS uses as separators)

If PIGOR does not ask these question on startup, they have already been configured. Either delete the \sphinxcode{\sphinxupquote{pigor.config}} file in the directory where \sphinxcode{\sphinxupquote{pigor.py}} is located, or just use the command {[}i{]} within PIGOR to reconfigure it.

Now you should see something like that, depending on PIGOR’s version:

\begin{sphinxVerbatim}[commandchars=\\\{\}]
\PYGZdl{} python pigor.py

\PYG{o}{=}\PYG{o}{=}\PYG{o}{=}\PYG{o}{=}\PYG{o}{=}\PYG{o}{=}\PYG{o}{=}\PYG{o}{=}\PYG{o}{=}\PYG{o}{=}\PYG{o}{=}\PYG{o}{=}\PYG{o}{=}\PYG{o}{=}\PYG{o}{=}\PYG{o}{=}\PYG{o}{=}\PYG{o}{=}\PYG{o}{=}\PYG{o}{=}\PYG{o}{=}\PYG{o}{=}
Welcome to PIGOR v1.1.
\PYG{o}{=}\PYG{o}{=}\PYG{o}{=}\PYG{o}{=}\PYG{o}{=}\PYG{o}{=}\PYG{o}{=}\PYG{o}{=}\PYG{o}{=}\PYG{o}{=}\PYG{o}{=}\PYG{o}{=}\PYG{o}{=}\PYG{o}{=}\PYG{o}{=}\PYG{o}{=}\PYG{o}{=}\PYG{o}{=}\PYG{o}{=}\PYG{o}{=}\PYG{o}{=}\PYG{o}{=}

i ... init
h ... print\PYGZus{}help
a ... analyse\PYGZus{}files
j ... create\PYGZus{}index
r ... remove\PYGZus{}generated\PYGZus{}files
x ... print\PYGZus{}root
q ... quit PIGOR v1.1

PIGOR v1.1 will look \PYG{k}{for} measurement files in D:\PYG{l+s+se}{\PYGZbs{}m}easurements.

If you need more information about a command, just \PYG{n+nb}{type} h + \PYG{o}{[}command\PYG{o}{]} + \PYGZlt{}ENTER\PYGZgt{}
to get more help. For example: h + a + \PYGZlt{}ENTER\PYGZgt{}.

Please \PYG{n+nb}{type} a \PYG{n+nb}{command} you want to perform and press \PYGZlt{}ENTER\PYGZgt{}.
\end{sphinxVerbatim}

On startup PIGOR is printing the root, which in this case is \sphinxcode{\sphinxupquote{D:\textbackslash{}measurements}}. It will also include a list of available commands. Each command can be triggered by pressing the corresponding letter followed by an \textless{}ENTER\textgreater{}. Some commands allow more options, each separated by a space.

We will describe these commands by {[}cmd{]} where cmd stands for the specific command.

To get more help on a specific command, just type {[}h{]} + {[}cmd{]}. For example typing \sphinxcode{\sphinxupquote{h a}} will give us the help on \sphinxcode{\sphinxupquote{analyse\_files()}}. Let’s try it:

\begin{sphinxVerbatim}[commandchars=\\\{\}]
Please \PYG{n+nb}{type} a \PYG{n+nb}{command} you want to perform and press \PYGZlt{}ENTER\PYGZgt{}.
h a
a:

Analyses all given files in list. This \PYG{k}{function} can be used by the \PYG{n+nb}{command} \PYG{o}{[}a\PYG{o}{]}.

    :param filepaths:   list of files to analyse with
                        their relative dir path added

    .. todo:: Change to no override mode. measurement.Measurement.plot\PYG{o}{(}\PYG{n+nv}{override}\PYG{o}{=}False\PYG{o}{)}
    .. todo:: a + \PYG{n+nv}{today} \PYG{o}{=}\PYGZgt{} only analyse files \PYG{k}{for} today
    .. todo:: a + \PYG{n+nv}{override} \PYG{o}{=}\PYGZgt{} \PYG{n+nv}{override}\PYG{o}{=}True
\end{sphinxVerbatim}

This is the same help text as found in this documentation.

We can now analyse our files with {[}a{]}. You can see on which file PIGOR is currently working on. If an error occurs, you will see it as well and PIGOR will skip the file.

After PIGOR is done analysing files, you may want to access these files. Use the {[}j{]} command to create an index to quickly go through all the files that have been analysed. Now you should see a new file has been created:
\begin{itemize}
\item {} \begin{description}
\item[{measurements}] \leavevmode\begin{itemize}
\item {} 
index\_pigor.html

\item {} 
2019-01

\item {} 
2019-02

\item {} 
2019-03

\end{itemize}

\end{description}

\end{itemize}

Open \sphinxcode{\sphinxupquote{index\_pigor.html}} to see the list. From there you can review the original data and the files that PIGOR created from this.


\chapter{PIGOR}
\label{\detokenize{pigor:pigor}}\label{\detokenize{pigor::doc}}

\section{What PIGOR does}
\label{\detokenize{pigor:what-pigor-does}}
PIGOR (‘Python IGOR’ = PIGOR) aims to help physicists on the NEPTUN beamline to quickly extract the needed information when measuring and configuring or preparing an experiment. It will go through all files in its root folder and will continue to \sphinxstylestrong{look for files in all subdirectories recursively} as well. It will then \sphinxstylestrong{auto detect} %
\begin{footnote}[1]\sphinxAtStartFootnote
This will only work if the correct naming convention is used.
%
\end{footnote} \sphinxstylestrong{the type of measurement} and guess what the user wants to know. After analysis of all files, \sphinxstylestrong{additional files will show up alongside the original measurement files}:
\begin{itemize}
\item {} 
.png file: plot of the data

\item {} 
.md file: usefull information gathered about the measurement in plain text as markdown

\item {} 
.html file: same content as the markdown file, but nicely viewable in a modern browser

\end{itemize}


\section{PIGORs inner workings}
\label{\detokenize{pigor:module-pigor}}\label{\detokenize{pigor:pigors-inner-workings}}\index{pigor (module)@\spxentry{pigor}\spxextra{module}}\index{analyse\_files() (in module pigor)@\spxentry{analyse\_files()}\spxextra{in module pigor}}

\begin{fulllineitems}
\phantomsection\label{\detokenize{pigor:pigor.analyse_files}}\pysiglinewithargsret{\sphinxcode{\sphinxupquote{pigor.}}\sphinxbfcode{\sphinxupquote{analyse\_files}}}{\emph{filepaths='all'}}{}
Analyses all given files in list. This function can be used by the command {[}a{]}.
\begin{quote}\begin{description}
\item[{Parameters}] \leavevmode
\sphinxstyleliteralstrong{\sphinxupquote{filepaths}} \textendash{} list of files to analyse with
their relative dir path added

\end{description}\end{quote}

\begin{sphinxadmonition}{note}{\label{pigor:index-0}Todo:}
Change to no override mode. measurement.Measurement.plot(override=False)
\end{sphinxadmonition}

\begin{sphinxadmonition}{note}{\label{pigor:index-1}Todo:}
a + today =\textgreater{} only analyse files for today
\end{sphinxadmonition}

\begin{sphinxadmonition}{note}{\label{pigor:index-2}Todo:}
a + override =\textgreater{} override=True
\end{sphinxadmonition}

\end{fulllineitems}

\index{bool2yn() (in module pigor)@\spxentry{bool2yn()}\spxextra{in module pigor}}

\begin{fulllineitems}
\phantomsection\label{\detokenize{pigor:pigor.bool2yn}}\pysiglinewithargsret{\sphinxcode{\sphinxupquote{pigor.}}\sphinxbfcode{\sphinxupquote{bool2yn}}}{\emph{b}}{}
Converts a boolean to yes or no with the mapping: y = True, n = False.

\end{fulllineitems}

\index{create\_index() (in module pigor)@\spxentry{create\_index()}\spxextra{in module pigor}}

\begin{fulllineitems}
\phantomsection\label{\detokenize{pigor:pigor.create_index}}\pysiglinewithargsret{\sphinxcode{\sphinxupquote{pigor.}}\sphinxbfcode{\sphinxupquote{create\_index}}}{}{}
Creates an index.html listing all directories and subdirectories and their HTML and Markdown files. This function can be used by the command {[}j{]}.

\end{fulllineitems}

\index{find\_all\_files() (in module pigor)@\spxentry{find\_all\_files()}\spxextra{in module pigor}}

\begin{fulllineitems}
\phantomsection\label{\detokenize{pigor:pigor.find_all_files}}\pysiglinewithargsret{\sphinxcode{\sphinxupquote{pigor.}}\sphinxbfcode{\sphinxupquote{find\_all\_files}}}{}{}
Finds all dat files recursively in all subdirectories ignoring hidden directories
and Python specific ones.

Returns a list of filepaths.

\end{fulllineitems}

\index{init() (in module pigor)@\spxentry{init()}\spxextra{in module pigor}}

\begin{fulllineitems}
\phantomsection\label{\detokenize{pigor:pigor.init}}\pysiglinewithargsret{\sphinxcode{\sphinxupquote{pigor.}}\sphinxbfcode{\sphinxupquote{init}}}{\emph{create\_new\_config\_file=True}}{}
This function will read the config file and initialize some variables
accordingly. This function can be used by the command {[}i{]}.

Available options:
\begin{itemize}
\item {} 
root directory where PIGOR will start to look for measurement files

\item {} 
Should PIGOR look for files to analyse recursively?

\item {} 
Which file extention do the measurement files posess?

\item {} 
What plot output format should PIGOR use?

\item {} 
Should PIGOR automatically create an html file?

\item {} 
Should PIGOR automatically create a md file?

\item {} 
Should PIGOR create a txt file containing all used fit functions for the use in Mathematica?

\end{itemize}

\begin{sphinxadmonition}{note}{Note:}
If no config file can be found, it will create one.
\end{sphinxadmonition}

\end{fulllineitems}

\index{is\_valid\_theme() (in module pigor)@\spxentry{is\_valid\_theme()}\spxextra{in module pigor}}

\begin{fulllineitems}
\phantomsection\label{\detokenize{pigor:pigor.is_valid_theme}}\pysiglinewithargsret{\sphinxcode{\sphinxupquote{pigor.}}\sphinxbfcode{\sphinxupquote{is\_valid\_theme}}}{\emph{theme}}{}
Checks if this theme exists.

\end{fulllineitems}

\index{list\_themes() (in module pigor)@\spxentry{list\_themes()}\spxextra{in module pigor}}

\begin{fulllineitems}
\phantomsection\label{\detokenize{pigor:pigor.list_themes}}\pysiglinewithargsret{\sphinxcode{\sphinxupquote{pigor.}}\sphinxbfcode{\sphinxupquote{list\_themes}}}{}{}
Returns a list of all themes available.

\end{fulllineitems}

\index{main() (in module pigor)@\spxentry{main()}\spxextra{in module pigor}}

\begin{fulllineitems}
\phantomsection\label{\detokenize{pigor:pigor.main}}\pysiglinewithargsret{\sphinxcode{\sphinxupquote{pigor.}}\sphinxbfcode{\sphinxupquote{main}}}{}{}
Main Loop

\end{fulllineitems}

\index{print\_header() (in module pigor)@\spxentry{print\_header()}\spxextra{in module pigor}}

\begin{fulllineitems}
\phantomsection\label{\detokenize{pigor:pigor.print_header}}\pysiglinewithargsret{\sphinxcode{\sphinxupquote{pigor.}}\sphinxbfcode{\sphinxupquote{print\_header}}}{\emph{text}}{}
This function prints a beautiful header followed by one empty line.
\begin{quote}\begin{description}
\item[{Parameters}] \leavevmode
\sphinxstyleliteralstrong{\sphinxupquote{text}} \textendash{} text to be displayed as header

\end{description}\end{quote}

\end{fulllineitems}

\index{print\_help() (in module pigor)@\spxentry{print\_help()}\spxextra{in module pigor}}

\begin{fulllineitems}
\phantomsection\label{\detokenize{pigor:pigor.print_help}}\pysiglinewithargsret{\sphinxcode{\sphinxupquote{pigor.}}\sphinxbfcode{\sphinxupquote{print\_help}}}{\emph{display='all'}}{}
Prints a help menu on the screen for the user. This function can be used by the command {[}h{]}.
\begin{quote}\begin{description}
\item[{Parameters}] \leavevmode
\sphinxstyleliteralstrong{\sphinxupquote{display}} \textendash{} specify the lenght of the help menu, options are ‘all’ or ‘quick’ (Default value = “all”)

\end{description}\end{quote}

\end{fulllineitems}

\index{print\_root() (in module pigor)@\spxentry{print\_root()}\spxextra{in module pigor}}

\begin{fulllineitems}
\phantomsection\label{\detokenize{pigor:pigor.print_root}}\pysiglinewithargsret{\sphinxcode{\sphinxupquote{pigor.}}\sphinxbfcode{\sphinxupquote{print\_root}}}{}{}
Prints the root for PIGOR, e.g. where it will look for files to analyse. This function
can be used by the command {[}x{]}.

\end{fulllineitems}

\index{remove\_generated\_files() (in module pigor)@\spxentry{remove\_generated\_files()}\spxextra{in module pigor}}

\begin{fulllineitems}
\phantomsection\label{\detokenize{pigor:pigor.remove_generated_files}}\pysiglinewithargsret{\sphinxcode{\sphinxupquote{pigor.}}\sphinxbfcode{\sphinxupquote{remove\_generated\_files}}}{\emph{files='all'}}{}
Removes the generated png, html and md files. This function can be used by the command {[}r{]}.
\begin{quote}\begin{description}
\item[{Parameters}] \leavevmode
\sphinxstyleliteralstrong{\sphinxupquote{files}} \textendash{} list of Path objects to files that should be removed; if set to ‘all’ it will delete all generated files (Default value = ‘all’)

\end{description}\end{quote}

\begin{sphinxadmonition}{note}{\label{pigor:index-3}Todo:}
Cover the case when files are not a list of path, e.g. wrong input given.
\end{sphinxadmonition}

\end{fulllineitems}

\index{show\_user() (in module pigor)@\spxentry{show\_user()}\spxextra{in module pigor}}

\begin{fulllineitems}
\phantomsection\label{\detokenize{pigor:pigor.show_user}}\pysiglinewithargsret{\sphinxcode{\sphinxupquote{pigor.}}\sphinxbfcode{\sphinxupquote{show\_user}}}{\emph{func}}{}
Register a function to be displayed to the user as an option

\end{fulllineitems}

\index{yn2bool() (in module pigor)@\spxentry{yn2bool()}\spxextra{in module pigor}}

\begin{fulllineitems}
\phantomsection\label{\detokenize{pigor:pigor.yn2bool}}\pysiglinewithargsret{\sphinxcode{\sphinxupquote{pigor.}}\sphinxbfcode{\sphinxupquote{yn2bool}}}{\emph{s}}{}
Converts yes and no to True and False.

\end{fulllineitems}



\chapter{The Measurement Class}
\label{\detokenize{measurement:the-measurement-class}}\label{\detokenize{measurement::doc}}

\section{Flow at Startup}
\label{\detokenize{measurement:flow-at-startup}}\begin{enumerate}
\def\theenumi{\arabic{enumi}}
\def\labelenumi{\theenumi .}
\makeatletter\def\p@enumii{\p@enumi \theenumi .}\makeatother
\item {} 
read the data

\item {} 
detect measurement type

\item {} 
read position file if a position file exists

\item {} 
clean up and description gathering

\item {} 
select columns for plotting

\end{enumerate}


\section{Flow when plotting}
\label{\detokenize{measurement:flow-when-plotting}}

\section{Class Usecases}
\label{\detokenize{measurement:class-usecases}}
There are many ways to interact with or use the Measurement class. Here are the three main ways:


\section{Methods}
\label{\detokenize{measurement:methods}}
\begin{sphinxadmonition}{note}{\label{measurement:index-0}Todo:}
Method attributes are shown, but value is always None.
\end{sphinxadmonition}
\phantomsection\label{\detokenize{measurement:module-measurement}}\index{measurement (module)@\spxentry{measurement}\spxextra{module}}\index{Measurement (class in measurement)@\spxentry{Measurement}\spxextra{class in measurement}}

\begin{fulllineitems}
\phantomsection\label{\detokenize{measurement:measurement.Measurement}}\pysiglinewithargsret{\sphinxbfcode{\sphinxupquote{class }}\sphinxcode{\sphinxupquote{measurement.}}\sphinxbfcode{\sphinxupquote{Measurement}}}{\emph{path}, \emph{type\_of\_measurement='default'}, \emph{type\_of\_fit='gauss'}}{}
This class provides an easy way to read, analyse and plot data from
text files.

There are two different file formats, which are used on the interferometry
as well as on the polarimeter station at Atominstitut of TU Wien. For more
information on the conventions please head to the docs or take a look at
the example files provided.
\index{FIT\_RESOLUTION (measurement.Measurement attribute)@\spxentry{FIT\_RESOLUTION}\spxextra{measurement.Measurement attribute}}

\begin{fulllineitems}
\phantomsection\label{\detokenize{measurement:measurement.Measurement.FIT_RESOLUTION}}\pysigline{\sphinxbfcode{\sphinxupquote{FIT\_RESOLUTION}}\sphinxbfcode{\sphinxupquote{ = None}}}
number of points to calculate the fit for

\end{fulllineitems}

\index{N\_HEADER (measurement.Measurement attribute)@\spxentry{N\_HEADER}\spxextra{measurement.Measurement attribute}}

\begin{fulllineitems}
\phantomsection\label{\detokenize{measurement:measurement.Measurement.N_HEADER}}\pysigline{\sphinxbfcode{\sphinxupquote{N\_HEADER}}\sphinxbfcode{\sphinxupquote{ = None}}}
number of lines of header of measurement file

\end{fulllineitems}

\index{\_\_init\_\_() (measurement.Measurement method)@\spxentry{\_\_init\_\_()}\spxextra{measurement.Measurement method}}

\begin{fulllineitems}
\phantomsection\label{\detokenize{measurement:measurement.Measurement.__init__}}\pysiglinewithargsret{\sphinxbfcode{\sphinxupquote{\_\_init\_\_}}}{\emph{path}, \emph{type\_of\_measurement='default'}, \emph{type\_of\_fit='gauss'}}{}
The Measurement class provides an easy and quick way to read, 
analyse and plot data from text files. When creating a new instance,
the following parameters have to be provided:
\begin{quote}
\begin{quote}\begin{description}
\item[{param self}] \leavevmode
the object itself

\item[{param path}] \leavevmode
pathlib.Path object

\item[{param type\_of\_measurement}] \leavevmode
used to hard set the type of measurement
on instance creation (default value = ‘default’)

\item[{param type\_of\_fit}] \leavevmode
sets an initial fit type, which may be overridden
by detect\_measurement\_type() later (default value
= ‘gauss’)
TODO: change to be permanent?

\end{description}\end{quote}
\end{quote}

The startup sequence is as follows:
\begin{enumerate}
\def\theenumi{\arabic{enumi}}
\def\labelenumi{\theenumi .}
\makeatletter\def\p@enumii{\p@enumi \theenumi .}\makeatother
\item {} 
try to read the data

\item {} 
measurement type (either set it, when given as input argument or try to detect it)

\item {} 
if measurement is POL, try to find a position file and read it

\item {} 
clean up the given data

\item {} 
select columns =\textgreater{} write into self.x and self.y

\item {} 
if measurement is POL, calculate degree of polarisation

\end{enumerate}

\begin{sphinxadmonition}{note}{\label{measurement:index-1}Todo:}
Is type\_of\_fit really needed?
\end{sphinxadmonition}

Returns nothing.

\end{fulllineitems}

\index{clean\_data() (measurement.Measurement method)@\spxentry{clean\_data()}\spxextra{measurement.Measurement method}}

\begin{fulllineitems}
\phantomsection\label{\detokenize{measurement:measurement.Measurement.clean_data}}\pysiglinewithargsret{\sphinxbfcode{\sphinxupquote{clean\_data}}}{}{}
Splits the \sphinxcode{\sphinxupquote{raw}} data into \sphinxcode{\sphinxupquote{head}} and \sphinxcode{\sphinxupquote{data}} vars.

\end{fulllineitems}

\index{contrast() (measurement.Measurement method)@\spxentry{contrast()}\spxextra{measurement.Measurement method}}

\begin{fulllineitems}
\phantomsection\label{\detokenize{measurement:measurement.Measurement.contrast}}\pysiglinewithargsret{\sphinxbfcode{\sphinxupquote{contrast}}}{\emph{source='fit'}}{}
Calculates the contrast of source as:

\sphinxcode{\sphinxupquote{contrast = (max-min) / (max+min)}}

where \sphinxcode{\sphinxupquote{min}} and \sphinxcode{\sphinxupquote{max}} are the minima and maxima of the given data.
\begin{quote}\begin{description}
\item[{Parameters}] \leavevmode
\sphinxstyleliteralstrong{\sphinxupquote{source}} \textendash{} defines the source of the data to calculate the contrast from,
can be either set to ‘fit’ or ‘data’ (Default value = ‘fit’)

\end{description}\end{quote}

Returns a list of contrasts.

\begin{sphinxadmonition}{note}{\label{measurement:index-2}Todo:}
When calculation of contrast fails, what should this function return? Now it returns {[}0{]}.
\end{sphinxadmonition}

\end{fulllineitems}

\index{degree\_of\_polarisation() (measurement.Measurement method)@\spxentry{degree\_of\_polarisation()}\spxextra{measurement.Measurement method}}

\begin{fulllineitems}
\phantomsection\label{\detokenize{measurement:measurement.Measurement.degree_of_polarisation}}\pysiglinewithargsret{\sphinxbfcode{\sphinxupquote{degree\_of\_polarisation}}}{}{}
Calculates the degree of polarisation for each position in \sphinxcode{\sphinxupquote{pos\_data}}.

\end{fulllineitems}

\index{detect\_measurement\_type() (measurement.Measurement method)@\spxentry{detect\_measurement\_type()}\spxextra{measurement.Measurement method}}

\begin{fulllineitems}
\phantomsection\label{\detokenize{measurement:measurement.Measurement.detect_measurement_type}}\pysiglinewithargsret{\sphinxbfcode{\sphinxupquote{detect\_measurement\_type}}}{}{}
This function auto detects the type of measurement based on the file
name. This works best with a meaningful file name convention. For more
information please refer to the docs.

Several \sphinxcode{\sphinxupquote{type\_of\_measurement}} can be detected:
\begin{itemize}
\item {} 
DC\#X: x-field of DC coil number \# scan -\textgreater{} sets \sphinxcode{\sphinxupquote{type\_of\_fit = 'sine\_lin'}}

\item {} 
DC\#Z: z-field of DC coil number \# scan -\textgreater{} sets \sphinxcode{\sphinxupquote{type\_of\_fit = 'poly5'}}

\item {} 
POS: scan of different linear stage positions -\textgreater{} sets \sphinxcode{\sphinxupquote{type\_of\_fit = 'gauss'}}

\end{itemize}

\sphinxcode{\sphinxupquote{type\_of\_fit}} can be overridden by explicitly mentioning a fit function to use
in the name of the file. See docs for more information.

In addition to the type of fit and measurement type, some additional
information about the measurement is gathered in the \sphinxcode{\sphinxupquote{settings}} dict.

\end{fulllineitems}

\index{export\_meta() (measurement.Measurement method)@\spxentry{export\_meta()}\spxextra{measurement.Measurement method}}

\begin{fulllineitems}
\phantomsection\label{\detokenize{measurement:measurement.Measurement.export_meta}}\pysiglinewithargsret{\sphinxbfcode{\sphinxupquote{export\_meta}}}{\emph{make\_md=True}, \emph{make\_html=False}, \emph{theme='github'}}{}
Exports all available information about the measurement into
a markdown file.
\begin{quote}\begin{description}
\item[{Parameters}] \leavevmode\begin{itemize}
\item {} 
\sphinxstyleliteralstrong{\sphinxupquote{html}} \textendash{} if set to True, an HTML file will be additionally
created (Default value = False)

\item {} 
\sphinxstyleliteralstrong{\sphinxupquote{theme}} \textendash{} set the default theme for html export, all
available themes can be found in the markdown\_themes
directory (Default value = ‘github’)

\end{itemize}

\end{description}\end{quote}

\end{fulllineitems}

\index{find\_bounds() (measurement.Measurement method)@\spxentry{find\_bounds()}\spxextra{measurement.Measurement method}}

\begin{fulllineitems}
\phantomsection\label{\detokenize{measurement:measurement.Measurement.find_bounds}}\pysiglinewithargsret{\sphinxbfcode{\sphinxupquote{find\_bounds}}}{\emph{fit\_function=None}}{}
Automatically finds usefull fit bounds and updates them
in the \sphinxcode{\sphinxupquote{fit\_function\_list}} dict.
\begin{quote}\begin{description}
\item[{Parameters}] \leavevmode
\sphinxstyleliteralstrong{\sphinxupquote{fit\_function}} \textendash{} defines for which fit functions the
bounds should be updated (Default
value = None), if set to None, type\_of\_fit
will be used

\end{description}\end{quote}

\end{fulllineitems}

\index{fit() (measurement.Measurement method)@\spxentry{fit()}\spxextra{measurement.Measurement method}}

\begin{fulllineitems}
\phantomsection\label{\detokenize{measurement:measurement.Measurement.fit}}\pysiglinewithargsret{\sphinxbfcode{\sphinxupquote{fit}}}{\emph{fit\_function=None}, \emph{fit\_function\_export=False}}{}
Fits the data in \sphinxcode{\sphinxupquote{x}} and \sphinxcode{\sphinxupquote{y}} using the default fit function of each
\sphinxcode{\sphinxupquote{type\_of\_fit}} if not specified further by passing a certain fit function as an
argument.
\begin{quote}\begin{description}
\item[{Parameters}] \leavevmode\begin{itemize}
\item {} 
\sphinxstyleliteralstrong{\sphinxupquote{fit\_function}} \textendash{} fit function to use to fit the data with (Default value = None)

\item {} 
\sphinxstyleliteralstrong{\sphinxupquote{fit\_function\_export}} \textendash{} exports the fit function as a txt file in a specified format (Mathematica is default and only implementation yet.).

\end{itemize}

\end{description}\end{quote}

Stores the optimal values and the covariances in \sphinxcode{\sphinxupquote{popt}} and \sphinxcode{\sphinxupquote{pcov}} for
later use.

\end{fulllineitems}

\index{fit\_function\_list (measurement.Measurement attribute)@\spxentry{fit\_function\_list}\spxextra{measurement.Measurement attribute}}

\begin{fulllineitems}
\phantomsection\label{\detokenize{measurement:measurement.Measurement.fit_function_list}}\pysigline{\sphinxbfcode{\sphinxupquote{fit\_function\_list}}\sphinxbfcode{\sphinxupquote{ = None}}}
list of fit functions that can be used; imported from fit\_functions.py

\end{fulllineitems}

\index{measurement\_type() (measurement.Measurement method)@\spxentry{measurement\_type()}\spxextra{measurement.Measurement method}}

\begin{fulllineitems}
\phantomsection\label{\detokenize{measurement:measurement.Measurement.measurement_type}}\pysiglinewithargsret{\sphinxbfcode{\sphinxupquote{measurement\_type}}}{\emph{type\_of\_measurement='default'}}{}
Sets the type of the measurement if parameter type\_of\_measurement is set.
\begin{quote}\begin{description}
\item[{Parameters}] \leavevmode\begin{itemize}
\item {} 
\sphinxstyleliteralstrong{\sphinxupquote{self}} \textendash{} object itself

\item {} 
\sphinxstyleliteralstrong{\sphinxupquote{type\_of\_measurement}} \textendash{} default”:   new type of measurement (default 
value = ‘default’)

\end{itemize}

\end{description}\end{quote}

Returns the current type of measurement.

\begin{sphinxadmonition}{note}{\label{measurement:index-3}Todo:}
Evaluate if this method (measurement\_type()) is needed at all.
\end{sphinxadmonition}

\begin{sphinxadmonition}{note}{\label{measurement:index-4}Todo:}
Set better default value for measurement type.
\end{sphinxadmonition}

\end{fulllineitems}

\index{path (measurement.Measurement attribute)@\spxentry{path}\spxextra{measurement.Measurement attribute}}

\begin{fulllineitems}
\phantomsection\label{\detokenize{measurement:measurement.Measurement.path}}\pysigline{\sphinxbfcode{\sphinxupquote{path}}\sphinxbfcode{\sphinxupquote{ = None}}}
path (pathlib.Path object) to the measurement file

\end{fulllineitems}

\index{plot() (measurement.Measurement method)@\spxentry{plot()}\spxextra{measurement.Measurement method}}

\begin{fulllineitems}
\phantomsection\label{\detokenize{measurement:measurement.Measurement.plot}}\pysiglinewithargsret{\sphinxbfcode{\sphinxupquote{plot}}}{\emph{column1=(0}, \emph{1)}, \emph{column2=(1}, \emph{1)}, \emph{fit=True}, \emph{type\_of\_plot=''}, \emph{override=True}, \emph{file\_extention='.png'}}{}
Creates a plot for the data. If fit is set to False the data fit won’t be
plotted, even if there exists one. Following parameters are possible:
\begin{quote}\begin{description}
\item[{Parameters}] \leavevmode\begin{itemize}
\item {} 
\sphinxstyleliteralstrong{\sphinxupquote{self}} \textendash{} the object itself

\item {} 
\sphinxstyleliteralstrong{\sphinxupquote{column1}} \textendash{} (column, nth element) to choose the data from for x-axis (Default value = (0)

\item {} 
\sphinxstyleliteralstrong{\sphinxupquote{column2}} \textendash{} (column, nth element) to choose the data from for y-axis (Default value = (1)

\item {} 
\sphinxstyleliteralstrong{\sphinxupquote{fit}} \textendash{} if set to False plotting of the fit will be supressed (Default value = True)

\item {} 
\sphinxstyleliteralstrong{\sphinxupquote{type\_of\_plot}} \textendash{} string to specify a certain plot type, which will be used
in the file name as well as in the plot title (Default value = ‘’)

\item {} 
\sphinxstyleliteralstrong{\sphinxupquote{override}} \textendash{} determines if plot image should be recreated if it already exists (Default value = True)

\end{itemize}

\end{description}\end{quote}

\begin{sphinxadmonition}{note}{\label{measurement:index-5}Todo:}
Make x and y labels more general, especially for interferometer files, where more that one y value list is needed.
\end{sphinxadmonition}

\end{fulllineitems}

\index{pos\_file\_path (measurement.Measurement attribute)@\spxentry{pos\_file\_path}\spxextra{measurement.Measurement attribute}}

\begin{fulllineitems}
\phantomsection\label{\detokenize{measurement:measurement.Measurement.pos_file_path}}\pysigline{\sphinxbfcode{\sphinxupquote{pos\_file\_path}}\sphinxbfcode{\sphinxupquote{ = None}}}
path (pathlib.Path object) to the corresponding position file

\end{fulllineitems}

\index{read\_data() (measurement.Measurement method)@\spxentry{read\_data()}\spxextra{measurement.Measurement method}}

\begin{fulllineitems}
\phantomsection\label{\detokenize{measurement:measurement.Measurement.read_data}}\pysiglinewithargsret{\sphinxbfcode{\sphinxupquote{read\_data}}}{\emph{path}}{}
Reads data from file and stores it in \sphinxcode{\sphinxupquote{raw}}.
\begin{quote}\begin{description}
\item[{Parameters}] \leavevmode\begin{itemize}
\item {} 
\sphinxstyleliteralstrong{\sphinxupquote{self}} \textendash{} the object itself

\item {} 
\sphinxstyleliteralstrong{\sphinxupquote{path}} \textendash{} a pathlib.Path object pointing to a measurement file

\end{itemize}

\end{description}\end{quote}

\end{fulllineitems}

\index{read\_pos\_file() (measurement.Measurement method)@\spxentry{read\_pos\_file()}\spxextra{measurement.Measurement method}}

\begin{fulllineitems}
\phantomsection\label{\detokenize{measurement:measurement.Measurement.read_pos_file}}\pysiglinewithargsret{\sphinxbfcode{\sphinxupquote{read\_pos\_file}}}{}{}
Looks for a position file and reads it into \sphinxcode{\sphinxupquote{pos\_data}}.

\begin{sphinxadmonition}{note}{\label{measurement:index-6}Todo:}
When searching for a position file, the lenght of the file should match. So it should be 1/4 of the size of the original measurement file.
\end{sphinxadmonition}

\end{fulllineitems}

\index{reset\_bounds() (measurement.Measurement method)@\spxentry{reset\_bounds()}\spxextra{measurement.Measurement method}}

\begin{fulllineitems}
\phantomsection\label{\detokenize{measurement:measurement.Measurement.reset_bounds}}\pysiglinewithargsret{\sphinxbfcode{\sphinxupquote{reset\_bounds}}}{\emph{fit\_function=None}}{}
Resets the bounds of the measurement type’s default fitting
function if not specified otherwise.

Reset values are \sphinxcode{\sphinxupquote{(-np.inf, np.inf)}}.
\begin{quote}\begin{description}
\item[{Parameters}] \leavevmode
\sphinxstyleliteralstrong{\sphinxupquote{fit\_function}} \textendash{} specifies the fit function for which the
bounds should be reset (Default value = None)

\end{description}\end{quote}

\end{fulllineitems}

\index{select\_columns() (measurement.Measurement method)@\spxentry{select\_columns()}\spxextra{measurement.Measurement method}}

\begin{fulllineitems}
\phantomsection\label{\detokenize{measurement:measurement.Measurement.select_columns}}\pysiglinewithargsret{\sphinxbfcode{\sphinxupquote{select\_columns}}}{\emph{m=None}}{}
Selects columns of the \sphinxcode{\sphinxupquote{data}} as specified in m (map) and
saves in \sphinxcode{\sphinxupquote{x}} and \sphinxcode{\sphinxupquote{y{[}{]}}}.

..note:: \sphinxcode{\sphinxupquote{y\_error{[}{]}}} is calculated as sqrt(y)
\begin{quote}\begin{description}
\item[{Parameters}] \leavevmode
\sphinxstyleliteralstrong{\sphinxupquote{m}} \textendash{} map, e.g. list of tuples or None values; if m=None
select\_columns will be skipped (Default value = None)

\end{description}\end{quote}

The map \sphinxcode{\sphinxupquote{m}} definies which columns of the original measurement
data will be used later. Only one x-axis can be defined, but multiple
y-axes may be used. The lenght of the map must not exceed the number
of the columns in \sphinxcode{\sphinxupquote{data}}, but can be less or equal.

Each map is a list of items, which can either be tuples or None
values, if a column should be skipped. In the case of a tuple, the
first value must be a string, either ‘x’ or ‘y’, which determines
if the column should be interpreted as an x- or y-axis. Its second
value describes what nth element of the columns should be selected.

A few examples:

\begin{sphinxVerbatim}[commandchars=\\\{\}]
\PYG{n}{m} \PYG{o}{=} \PYG{p}{[}\PYG{p}{(}\PYG{l+s+s1}{\PYGZsq{}}\PYG{l+s+s1}{x}\PYG{l+s+s1}{\PYGZsq{}}\PYG{p}{,}\PYG{l+m+mi}{1}\PYG{p}{)}\PYG{p}{,}\PYG{p}{(}\PYG{l+s+s1}{\PYGZsq{}}\PYG{l+s+s1}{y}\PYG{l+s+s1}{\PYGZsq{}}\PYG{p}{,}\PYG{l+m+mi}{1}\PYG{p}{)}\PYG{p}{]}
\end{sphinxVerbatim}

This will select the first column as x-axis and take every (1st)
element of it, and the second column as y-axis, also using every
element of that column.

\begin{sphinxVerbatim}[commandchars=\\\{\}]
\PYG{n}{m} \PYG{o}{=} \PYG{p}{[}\PYG{p}{(}\PYG{l+s+s1}{\PYGZsq{}}\PYG{l+s+s1}{y}\PYG{l+s+s1}{\PYGZsq{}}\PYG{p}{,}\PYG{l+m+mi}{2}\PYG{p}{)}\PYG{p}{,}\PYG{k+kc}{None}\PYG{p}{,}\PYG{p}{(}\PYG{l+s+s1}{\PYGZsq{}}\PYG{l+s+s1}{x}\PYG{l+s+s1}{\PYGZsq{}}\PYG{p}{,}\PYG{l+m+mi}{2}\PYG{p}{)}\PYG{p}{,}\PYG{p}{(}\PYG{l+s+s1}{\PYGZsq{}}\PYG{l+s+s1}{y}\PYG{l+s+s1}{\PYGZsq{}}\PYG{p}{,}\PYG{l+m+mi}{2}\PYG{p}{)}\PYG{p}{]}
\end{sphinxVerbatim}

Here we will take every second element of column 1, 3 and 4, but
skip column 2.

\begin{sphinxadmonition}{note}{Note:}
If the lenght of the map is less than the number of columns in \sphinxcode{\sphinxupquote{data}}, every column that has no corresponding map element will be skipped.
\end{sphinxadmonition}

\end{fulllineitems}

\index{settings (measurement.Measurement attribute)@\spxentry{settings}\spxextra{measurement.Measurement attribute}}

\begin{fulllineitems}
\phantomsection\label{\detokenize{measurement:measurement.Measurement.settings}}\pysigline{\sphinxbfcode{\sphinxupquote{settings}}\sphinxbfcode{\sphinxupquote{ = None}}}
dict containing useful information read from files header in clean\_data()

\end{fulllineitems}

\index{type\_of\_fit (measurement.Measurement attribute)@\spxentry{type\_of\_fit}\spxextra{measurement.Measurement attribute}}

\begin{fulllineitems}
\phantomsection\label{\detokenize{measurement:measurement.Measurement.type_of_fit}}\pysigline{\sphinxbfcode{\sphinxupquote{type\_of\_fit}}\sphinxbfcode{\sphinxupquote{ = None}}}
type of fit to be applied to the data

\end{fulllineitems}


\end{fulllineitems}



\chapter{Measurement Class 2.0}
\label{\detokenize{measurement2:measurement-class-2-0}}\label{\detokenize{measurement2::doc}}
Due to some design flaws in the existsing {\hyperref[\detokenize{measurement:measurement.Measurement}]{\sphinxcrossref{\sphinxcode{\sphinxupquote{measurement.Measurement}}}}} class, it will be rewritten from ground up using already existsing libraries. This undertaking will lead to a cleaner version and will make it possible to adapt {\hyperref[\detokenize{measurement:measurement.Measurement}]{\sphinxcrossref{\sphinxcode{\sphinxupquote{measurement.Measurement}}}}} more easily for the interferometer experiments.


\section{Structure}
\label{\detokenize{measurement2:structure}}
The new immproved structure will make use of:
\begin{itemize}
\item {} 
LMFIT package

\item {} 
pandas dataframe

\end{itemize}

So that the fitting will be done with LMFIT models, whereas the data is handled in pandas dataframes. This creates huge advantages for the developer as well as for the user.


\section{Previous Development}
\label{\detokenize{measurement2:previous-development}}
There have been efforts previously to build a unique own modular {\hyperref[\detokenize{measurement:measurement.Measurement}]{\sphinxcrossref{\sphinxcode{\sphinxupquote{measurement.Measurement}}}}} class by Nico. These efforts can be examined in the branch improvements/core.

These following elements were attemted to build:
\begin{itemize}
\item {} 
data column: has data and a head, knows its name etc.; functions can easily be applied to it

\item {} 
fit object: used for fitting and finding bounds; each instance can have its own bounds

\item {} \begin{description}
\item[{data set: these objects can be plotted by Measurement, so Measurement will try to create one of those objects; they consists of:}] \leavevmode\begin{itemize}
\item {} 
data columns objects

\item {} 
fit objects

\end{itemize}

\end{description}

\end{itemize}

\sphinxstylestrong{Column Class}

Methods:
\begin{itemize}
\item {} 
\sphinxcode{\sphinxupquote{reverse()}}: reverse order of data

\item {} 
\sphinxcode{\sphinxupquote{\_\_init\_\_(self, desc, data)}}

\item {} 
\sphinxcode{\sphinxupquote{\_\_repr\_\_()}}: plots \sphinxcode{\sphinxupquote{'\textless{}column object 'desc' of lenght len(data)\textgreater{}'}} or something like that

\end{itemize}

Variables:
\begin{itemize}
\item {} 
\sphinxcode{\sphinxupquote{columns.data}}: holds the data as numpy array in float64

\item {} 
\sphinxcode{\sphinxupquote{columns.desc}}: holds the name of the columns heading as string

\end{itemize}

\sphinxstylestrong{Fit Function Class}

Method:
\begin{itemize}
\item {} 
\sphinxcode{\sphinxupquote{fit\_function()}}

\item {} 
\sphinxcode{\sphinxupquote{find\_bounds()}}: tries to find the bounds

\item {} 
\sphinxcode{\sphinxupquote{bounds}}: holds the bounds to be used when fitting as array of tuples

\end{itemize}

Variables:
\begin{itemize}
\item {} 
\sphinxcode{\sphinxupquote{parameters}}: dictionary holding the names of the parameters and the parameters themselves

\end{itemize}

\sphinxstylestrong{Fit Class}

Variables:
\begin{itemize}
\item {} 
\sphinxcode{\sphinxupquote{type}} with which function the fit should be carried out, string

\item {} 
\sphinxcode{\sphinxupquote{popt}}

\item {} 
\sphinxcode{\sphinxupquote{pcov}}

\end{itemize}

Methods:
\begin{itemize}
\item {} 
\sphinxcode{\sphinxupquote{fit()}}

\end{itemize}

\sphinxstylestrong{Data Set Class}

This object stores data points (lists or Column objects) to form a data set. It must contain at least two data point lists. These lists must have the same number of elements, if not the lists that don’t have enough elements will get padded with 0.

Variables:
\begin{itemize}
\item {} 
\sphinxcode{\sphinxupquote{desc}}: a description of what the data set describes (optional)

\item {} 
\sphinxcode{\sphinxupquote{data}}: data is stored in list; len(list) \textgreater{} 1;

\end{itemize}

It is obvious to every reader that indeed almost all of this functionality is already included in the python package pandas and lmfit. This led to the conclusion that improvements/core won’t be maintained any longer.


\section{The brand new Measurement Class}
\label{\detokenize{measurement2:the-brand-new-measurement-class}}
The brand new measurement class will include only the following featues:
\begin{itemize}
\item {} 
ready-to-use lmfit fit models

\item {} 
fit() method to actually produce a fit using lmfits minimize()

\item {} 
plot() method for implementing plotting functionality

\item {} 
contrast() to calculate the contrast

\item {} 
degree\_of\_polarisation() to calculate the degree of polarisation

\item {} 
read() to read data in a very generic way

\item {} 
export\_meta() to export meta data

\end{itemize}

Its child classes will then implement the experiment specific data handling details like cleaning the data and extending the meta data that will be exported.


\chapter{Fit Functions}
\label{\detokenize{fit-functions:fit-functions}}\label{\detokenize{fit-functions::doc}}
The following fit functions are implemented and can be used within PIGOR or the Measurement class. They can be found in \sphinxcode{\sphinxupquote{fit\_functions.py}}.

\phantomsection\label{\detokenize{fit-functions:module-fit_functions}}\index{fit\_functions (module)@\spxentry{fit\_functions}\spxextra{module}}\index{gauss() (in module fit\_functions)@\spxentry{gauss()}\spxextra{in module fit\_functions}}

\begin{fulllineitems}
\phantomsection\label{\detokenize{fit-functions:fit_functions.gauss}}\pysiglinewithargsret{\sphinxcode{\sphinxupquote{fit\_functions.}}\sphinxbfcode{\sphinxupquote{gauss}}}{\emph{x}, \emph{a}, \emph{x0}, \emph{sigma}, \emph{export=False}}{}
Gaussian function, used for fitting data.
\begin{quote}\begin{description}
\item[{Parameters}] \leavevmode\begin{itemize}
\item {} 
\sphinxstyleliteralstrong{\sphinxupquote{x}} \textendash{} parameter

\item {} 
\sphinxstyleliteralstrong{\sphinxupquote{a}} \textendash{} amplitude

\item {} 
\sphinxstyleliteralstrong{\sphinxupquote{x0}} \textendash{} maximum

\item {} 
\sphinxstyleliteralstrong{\sphinxupquote{sigma}} \textendash{} width

\item {} 
\sphinxstyleliteralstrong{\sphinxupquote{export}} \textendash{} enable text output of function

\end{itemize}

\end{description}\end{quote}

\end{fulllineitems}

\index{poly() (in module fit\_functions)@\spxentry{poly()}\spxextra{in module fit\_functions}}

\begin{fulllineitems}
\phantomsection\label{\detokenize{fit-functions:fit_functions.poly}}\pysiglinewithargsret{\sphinxcode{\sphinxupquote{fit\_functions.}}\sphinxbfcode{\sphinxupquote{poly}}}{\emph{x}, \emph{*args}, \emph{export=False}}{}
Polynom nth degree for fitting.
\begin{quote}\begin{description}
\item[{Parameters}] \leavevmode\begin{itemize}
\item {} 
\sphinxstyleliteralstrong{\sphinxupquote{x}} (\sphinxstyleliteralemphasis{\sphinxupquote{int}}\sphinxstyleliteralemphasis{\sphinxupquote{, }}\sphinxstyleliteralemphasis{\sphinxupquote{float}}) \textendash{} parameter

\item {} 
\sphinxstyleliteralstrong{\sphinxupquote{*args}} \textendash{} 
list of coefficients {[}a\_N,a\_N-1, …, a\_1, a\_0{]}


\item {} 
\sphinxstyleliteralstrong{\sphinxupquote{export}} (\sphinxstyleliteralemphasis{\sphinxupquote{bool}}\sphinxstyleliteralemphasis{\sphinxupquote{ or }}\sphinxstyleliteralemphasis{\sphinxupquote{string}}\sphinxstyleliteralemphasis{\sphinxupquote{, }}\sphinxstyleliteralemphasis{\sphinxupquote{optional}}) \textendash{} enable text output of function, defaults to False

\end{itemize}

\item[{Returns}] \leavevmode
returns the polynomial

\item[{Return type}] \leavevmode
str, int, float

\end{description}\end{quote}

\begin{sphinxVerbatim}[commandchars=\\\{\}]
\PYG{g+gp}{\PYGZgt{}\PYGZgt{}\PYGZgt{} }\PYG{n}{poly}\PYG{p}{(}\PYG{l+m+mf}{3.4543}\PYG{p}{,} \PYG{l+m+mi}{5}\PYG{p}{,}\PYG{l+m+mi}{4}\PYG{p}{,}\PYG{l+m+mi}{3}\PYG{p}{,}\PYG{l+m+mi}{2}\PYG{p}{,}\PYG{l+m+mi}{1}\PYG{p}{,} \PYG{n}{export}\PYG{o}{=}\PYG{l+s+s1}{\PYGZsq{}}\PYG{l+s+s1}{Mathematica}\PYG{l+s+s1}{\PYGZsq{}}\PYG{p}{)}
\PYG{g+go}{\PYGZsq{}5*3.4543\PYGZca{}5 + 4*3.4543\PYGZca{}4 + 3*3.4543\PYGZca{}3 + 2*3.4543\PYGZca{}2 + 1*3.4543\PYGZca{}1\PYGZsq{}}
\end{sphinxVerbatim}

\begin{sphinxVerbatim}[commandchars=\\\{\}]
\PYG{g+gp}{\PYGZgt{}\PYGZgt{}\PYGZgt{} }\PYG{n}{poly}\PYG{p}{(}\PYG{l+m+mf}{3.4543}\PYG{p}{,} \PYG{l+m+mi}{5}\PYG{p}{,}\PYG{l+m+mi}{4}\PYG{p}{,}\PYG{l+m+mi}{3}\PYG{p}{,}\PYG{l+m+mi}{2}\PYG{p}{,}\PYG{l+m+mi}{1}\PYG{p}{)}
\PYG{g+go}{920.4602110784704}
\end{sphinxVerbatim}

\end{fulllineitems}

\index{poly5() (in module fit\_functions)@\spxentry{poly5()}\spxextra{in module fit\_functions}}

\begin{fulllineitems}
\phantomsection\label{\detokenize{fit-functions:fit_functions.poly5}}\pysiglinewithargsret{\sphinxcode{\sphinxupquote{fit\_functions.}}\sphinxbfcode{\sphinxupquote{poly5}}}{\emph{x}, \emph{a5}, \emph{a4}, \emph{a3}, \emph{a2}, \emph{a1}, \emph{a0}, \emph{export=False}}{}
Polynom 5th degree for fitting.
\begin{quote}\begin{description}
\item[{Parameters}] \leavevmode\begin{itemize}
\item {} 
\sphinxstyleliteralstrong{\sphinxupquote{x}} \textendash{} parameter

\item {} 
\sphinxstyleliteralstrong{\sphinxupquote{a5}} \textendash{} coeff

\item {} 
\sphinxstyleliteralstrong{\sphinxupquote{a4}} \textendash{} coeff

\item {} 
\sphinxstyleliteralstrong{\sphinxupquote{a3}} \textendash{} coeff

\item {} 
\sphinxstyleliteralstrong{\sphinxupquote{a2}} \textendash{} coeff

\item {} 
\sphinxstyleliteralstrong{\sphinxupquote{a1}} \textendash{} coeff

\item {} 
\sphinxstyleliteralstrong{\sphinxupquote{a0}} \textendash{} coeff

\item {} 
\sphinxstyleliteralstrong{\sphinxupquote{export}} \textendash{} enable text output of function

\end{itemize}

\item[{Returns}] \leavevmode
function \textendash{} polynomial 5th degree

\end{description}\end{quote}

\end{fulllineitems}

\index{register\_fit\_function() (in module fit\_functions)@\spxentry{register\_fit\_function()}\spxextra{in module fit\_functions}}

\begin{fulllineitems}
\phantomsection\label{\detokenize{fit-functions:fit_functions.register_fit_function}}\pysiglinewithargsret{\sphinxcode{\sphinxupquote{fit\_functions.}}\sphinxbfcode{\sphinxupquote{register\_fit\_function}}}{\emph{func}, \emph{bounds=(-inf}, \emph{inf)}}{}
This decorator registers a new fit function and writes an entry to fit\_function\_list.

\end{fulllineitems}

\index{sine() (in module fit\_functions)@\spxentry{sine()}\spxextra{in module fit\_functions}}

\begin{fulllineitems}
\phantomsection\label{\detokenize{fit-functions:fit_functions.sine}}\pysiglinewithargsret{\sphinxcode{\sphinxupquote{fit\_functions.}}\sphinxbfcode{\sphinxupquote{sine}}}{\emph{x}, \emph{a}, \emph{omega}, \emph{phase}, \emph{c}, \emph{export=False}}{}
Sine function for fitting data.
\begin{quote}\begin{description}
\item[{Parameters}] \leavevmode\begin{itemize}
\item {} 
\sphinxstyleliteralstrong{\sphinxupquote{x}} \textendash{} parameter

\item {} 
\sphinxstyleliteralstrong{\sphinxupquote{a}} \textendash{} amplitude

\item {} 
\sphinxstyleliteralstrong{\sphinxupquote{omega}} \textendash{} frequency

\item {} 
\sphinxstyleliteralstrong{\sphinxupquote{phase}} \textendash{} phase

\item {} 
\sphinxstyleliteralstrong{\sphinxupquote{c}} \textendash{} offset

\item {} 
\sphinxstyleliteralstrong{\sphinxupquote{export}} \textendash{} enable text output of function

\end{itemize}

\end{description}\end{quote}

\end{fulllineitems}

\index{sine\_lin() (in module fit\_functions)@\spxentry{sine\_lin()}\spxextra{in module fit\_functions}}

\begin{fulllineitems}
\phantomsection\label{\detokenize{fit-functions:fit_functions.sine_lin}}\pysiglinewithargsret{\sphinxcode{\sphinxupquote{fit\_functions.}}\sphinxbfcode{\sphinxupquote{sine\_lin}}}{\emph{x}, \emph{a}, \emph{omega}, \emph{phase}, \emph{c}, \emph{b}, \emph{export=False}}{}
Sine function with linear term added for fitting data.
\begin{quote}\begin{description}
\item[{Parameters}] \leavevmode\begin{itemize}
\item {} 
\sphinxstyleliteralstrong{\sphinxupquote{x}} \textendash{} parameter

\item {} 
\sphinxstyleliteralstrong{\sphinxupquote{a}} \textendash{} amplitude

\item {} 
\sphinxstyleliteralstrong{\sphinxupquote{omega}} \textendash{} frequency

\item {} 
\sphinxstyleliteralstrong{\sphinxupquote{phase}} \textendash{} phase

\item {} 
\sphinxstyleliteralstrong{\sphinxupquote{c}} \textendash{} offset

\item {} 
\sphinxstyleliteralstrong{\sphinxupquote{b}} \textendash{} slope

\item {} 
\sphinxstyleliteralstrong{\sphinxupquote{export}} \textendash{} enable text output of function

\end{itemize}

\end{description}\end{quote}

\end{fulllineitems}



\chapter{Sprint Planning}
\label{\detokenize{sprints:sprint-planning}}\label{\detokenize{sprints::doc}}
This site gives a quick overview what will come next. Each sprint should take about 1 week to finish.


\section{PIGOR}
\label{\detokenize{sprints:pigor}}\begin{enumerate}
\def\theenumi{\arabic{enumi}}
\def\labelenumi{\theenumi .}
\makeatletter\def\p@enumii{\p@enumi \theenumi .}\makeatother
\item {} \begin{description}
\item[{Sprint}] \leavevmode\begin{itemize}
\item {} 
\(\checkmark\) feature: remove all generated files (html, md, png)

\item {} 
\(\checkmark\) feature: introducing a config file (PIGOR start directory, …)

\item {} 
\(\checkmark\) improvement: auto create config file if not present

\item {} 
\(\checkmark\) improvement: auto register all functions for help menu (decorators)

\end{itemize}

\end{description}

\item {} \begin{description}
\item[{Sprint}] \leavevmode\begin{itemize}
\item {} 
feature: remove last generated files (html, md, png)

\item {} 
feature: remove all html/md or png files

\item {} 
\(\checkmark\) improvement: use JSON for config file

\end{itemize}

\end{description}

\item {} \begin{description}
\item[{Sprint:}] \leavevmode\begin{itemize}
\item {} 
feature: auto run command in specified intervals; syntax maybe: time + {[}cmd{]} + \textless{}ENTER\textgreater{}

\end{itemize}

\end{description}

\end{enumerate}


\section{Measurement Class}
\label{\detokenize{sprints:measurement-class}}\begin{enumerate}
\def\theenumi{\arabic{enumi}}
\def\labelenumi{\theenumi .}
\makeatletter\def\p@enumii{\p@enumi \theenumi .}\makeatother
\item {} \begin{description}
\item[{Sprint}] \leavevmode\begin{itemize}
\item {} 
\(\checkmark\) improvement: switching from self.y \textendash{}\textgreater{} self.y{[}{]} and self.y\_error \textendash{}\textgreater{} self.y\_error{[}{]}

\item {} 
\(\checkmark\) {[}not tested yet{]} improvement: plot multiple self.y’s

\item {} 
feature: auto detect interferometer measurements

\end{itemize}

\end{description}

\item {} \begin{description}
\item[{Sprint}] \leavevmode\begin{itemize}
\item {} 
feature: remove all associated files from file system, except the measurement file itself

\item {} 
\(\checkmark\) improvement: auto register all available fit functions via decorators

\item {} 
improvement: adding \_\_repr\_\_

\end{itemize}

\end{description}

\item {} \begin{description}
\item[{Sprint: finish branch \sphinxcode{\sphinxupquote{feature/interferometer}}}] \leavevmode\begin{itemize}
\item {} 
fixing / understanding inheritance of instance variables (see python test file in branch)

\item {} \begin{description}
\item[{creating subclasses from Measurement:}] \leavevmode\begin{itemize}
\item {} 
Interferometer: adding custom maps to COLUMN\_MAPS and overriding clean\_data() and detect\_measurement()

\item {} 
Polarimeter: adding custom maps to COLUMN\_MAPS and overriding clean\_data() and detect\_measurement()

\end{itemize}

\end{description}

\end{itemize}

\end{description}

\item {} 
Sprint: not planned yet

\end{enumerate}


\section{Ideas}
\label{\detokenize{sprints:ideas}}
Building Measurement from ground up with custom objects like:
\begin{itemize}
\item {} 
data column: has data and a head, knows its name etc.; functions can easily be applied to it

\item {} 
fit object: used for fitting and finding bounds; each instance can have its own bounds

\item {} \begin{description}
\item[{data set: these objects can be plotted by Measurement, so Measurement will try to create one of those objects; they consists of:}] \leavevmode\begin{itemize}
\item {} 
data columns objects

\item {} 
fit objects

\end{itemize}

\end{description}

\end{itemize}


\section{Column Class}
\label{\detokenize{sprints:column-class}}
Methods:
\begin{itemize}
\item {} 
\sphinxcode{\sphinxupquote{reverse()}}: reverse order of data

\item {} 
\sphinxcode{\sphinxupquote{\_\_init\_\_(self, desc, data)}}

\item {} 
\sphinxcode{\sphinxupquote{\_\_repr\_\_()}}: plots \sphinxcode{\sphinxupquote{'\textless{}column object 'desc' of lenght len(data)\textgreater{}'}} or something like that

\end{itemize}

Variables:
\begin{itemize}
\item {} 
\sphinxcode{\sphinxupquote{columns.data}}: holds the data as numpy array in float64

\item {} 
\sphinxcode{\sphinxupquote{columns.desc}}: holds the name of the columns heading as string

\end{itemize}


\section{Fit Function Class}
\label{\detokenize{sprints:fit-function-class}}
Method:
\begin{itemize}
\item {} 
\sphinxcode{\sphinxupquote{fit\_function()}}

\item {} 
\sphinxcode{\sphinxupquote{find\_bounds()}}: tries to find the bounds

\item {} 
\sphinxcode{\sphinxupquote{bounds}}: holds the bounds to be used when fitting as array of tuples

\end{itemize}

Variables:
\begin{itemize}
\item {} 
\sphinxcode{\sphinxupquote{parameters}}: dictionary holding the names of the parameters and the parameters themselves

\end{itemize}


\section{Fit Class}
\label{\detokenize{sprints:fit-class}}
Variables:
\begin{itemize}
\item {} 
\sphinxcode{\sphinxupquote{type}} with which function the fit should be carried out, string

\item {} 
\sphinxcode{\sphinxupquote{popt}}

\item {} 
\sphinxcode{\sphinxupquote{pcov}}

\end{itemize}

Methods:
\begin{itemize}
\item {} 
\sphinxcode{\sphinxupquote{fit()}}

\end{itemize}


\section{Data Set Class}
\label{\detokenize{sprints:data-set-class}}
This object stores data points (lists or Column objects) to form a data set. It must contain at least two data point lists. These lists must have the same number of elements, if not the lists that don’t have enough elements will get padded with 0.

Variables:
\begin{itemize}
\item {} 
\sphinxcode{\sphinxupquote{desc}}: a description of what the data set describes (optional)

\item {} 
\sphinxcode{\sphinxupquote{data}}: data is stored in list; len(list) \textgreater{} 1;

\end{itemize}


\chapter{ToDo List}
\label{\detokenize{todo:todo-list}}\label{\detokenize{todo::doc}}
\begin{sphinxadmonition}{note}{Todo:}
check if dependencies are correct with dependencies.txt
\end{sphinxadmonition}

(The {\hyperref[\detokenize{index:index-0}]{\sphinxcrossref{\sphinxstyleemphasis{original entry}}}} is located in /Users/nicoeinsidler/Code/pigor/doc/index.rst, line 55.)

\begin{sphinxadmonition}{note}{Todo:}
Method attributes are shown, but value is always None.
\end{sphinxadmonition}

(The {\hyperref[\detokenize{measurement:index-0}]{\sphinxcrossref{\sphinxstyleemphasis{original entry}}}} is located in /Users/nicoeinsidler/Code/pigor/doc/measurement.rst, line 62.)

\begin{sphinxadmonition}{note}{Todo:}
Is type\_of\_fit really needed?
\end{sphinxadmonition}

(The {\hyperref[\detokenize{measurement:index-1}]{\sphinxcrossref{\sphinxstyleemphasis{original entry}}}} is located in /Users/nicoeinsidler/Code/pigor/measurement.py:docstring of measurement.Measurement.\_\_init\_\_, line 23.)

\begin{sphinxadmonition}{note}{Todo:}
When calculation of contrast fails, what should this function return? Now it returns {[}0{]}.
\end{sphinxadmonition}

(The {\hyperref[\detokenize{measurement:index-2}]{\sphinxcrossref{\sphinxstyleemphasis{original entry}}}} is located in /Users/nicoeinsidler/Code/pigor/measurement.py:docstring of measurement.Measurement.contrast, line 12.)

\begin{sphinxadmonition}{note}{Todo:}
Evaluate if this method (measurement\_type()) is needed at all.
\end{sphinxadmonition}

(The {\hyperref[\detokenize{measurement:index-3}]{\sphinxcrossref{\sphinxstyleemphasis{original entry}}}} is located in /Users/nicoeinsidler/Code/pigor/measurement.py:docstring of measurement.Measurement.measurement\_type, line 9.)

\begin{sphinxadmonition}{note}{Todo:}
Set better default value for measurement type.
\end{sphinxadmonition}

(The {\hyperref[\detokenize{measurement:index-4}]{\sphinxcrossref{\sphinxstyleemphasis{original entry}}}} is located in /Users/nicoeinsidler/Code/pigor/measurement.py:docstring of measurement.Measurement.measurement\_type, line 10.)

\begin{sphinxadmonition}{note}{Todo:}
Make x and y labels more general, especially for interferometer files, where more that one y value list is needed.
\end{sphinxadmonition}

(The {\hyperref[\detokenize{measurement:index-5}]{\sphinxcrossref{\sphinxstyleemphasis{original entry}}}} is located in /Users/nicoeinsidler/Code/pigor/measurement.py:docstring of measurement.Measurement.plot, line 12.)

\begin{sphinxadmonition}{note}{Todo:}
When searching for a position file, the lenght of the file should match. So it should be 1/4 of the size of the original measurement file.
\end{sphinxadmonition}

(The {\hyperref[\detokenize{measurement:index-6}]{\sphinxcrossref{\sphinxstyleemphasis{original entry}}}} is located in /Users/nicoeinsidler/Code/pigor/measurement.py:docstring of measurement.Measurement.read\_pos\_file, line 3.)

\begin{sphinxadmonition}{note}{Todo:}
Change to no override mode. measurement.Measurement.plot(override=False)
\end{sphinxadmonition}

(The {\hyperref[\detokenize{pigor:index-0}]{\sphinxcrossref{\sphinxstyleemphasis{original entry}}}} is located in /Users/nicoeinsidler/Code/pigor/pigor.py:docstring of pigor.analyse\_files, line 6.)

\begin{sphinxadmonition}{note}{Todo:}
a + today =\textgreater{} only analyse files for today
\end{sphinxadmonition}

(The {\hyperref[\detokenize{pigor:index-1}]{\sphinxcrossref{\sphinxstyleemphasis{original entry}}}} is located in /Users/nicoeinsidler/Code/pigor/pigor.py:docstring of pigor.analyse\_files, line 7.)

\begin{sphinxadmonition}{note}{Todo:}
a + override =\textgreater{} override=True
\end{sphinxadmonition}

(The {\hyperref[\detokenize{pigor:index-2}]{\sphinxcrossref{\sphinxstyleemphasis{original entry}}}} is located in /Users/nicoeinsidler/Code/pigor/pigor.py:docstring of pigor.analyse\_files, line 8.)

\begin{sphinxadmonition}{note}{Todo:}
Cover the case when files are not a list of path, e.g. wrong input given.
\end{sphinxadmonition}

(The {\hyperref[\detokenize{pigor:index-3}]{\sphinxcrossref{\sphinxstyleemphasis{original entry}}}} is located in /Users/nicoeinsidler/Code/pigor/pigor.py:docstring of pigor.remove\_generated\_files, line 5.)


\chapter{Old ToDos}
\label{\detokenize{index:old-todos}}
\begin{sphinxadmonition}{note}{Note:}
This list of ToDos should be integrated into the docstrings or into the sprint planning where those ToDos belong.
\end{sphinxadmonition}

Here are all ToDos listed. Feel free to contribute and check this project out on Bitbucket.
\begin{itemize}
\item {} 
self.x\_error: not yet implemented

\item {} 
a lot of commenting

\item {} 
error of fit

\item {} 
verbose mode on/off

\item {} 
getting PIGOR ready for shipment by creating a setup.py

\item {} 
better display of self.pcov

\item {} 
separation between pure functions and functions with context (methods)

\item {} 
auto comment decorator for functions

\item {} 
use decorators to auto register fit functions with their input argument list

\item {} 
setuptools in setup.py

\item {} 
\sphinxurl{https://click.palletsprojects.com/en/7.x/quickstart/}

\end{itemize}


\chapter{Project Dependencies}
\label{\detokenize{index:project-dependencies}}
\begin{sphinxadmonition}{note}{\label{index:index-0}Todo:}
check if dependencies are correct with dependencies.txt
\end{sphinxadmonition}
\begin{itemize}
\item {} 
numpy

\item {} 
re

\item {} 
matplotlib

\item {} 
os.path

\item {} 
glob

\item {} 
difflib

\item {} 
datetime

\item {} 
pathlib

\item {} 
scipy

\item {} 
markdown

\end{itemize}


\chapter{Indices and tables}
\label{\detokenize{index:indices-and-tables}}\begin{itemize}
\item {} 
\DUrole{xref,std,std-ref}{genindex}

\item {} 
\DUrole{xref,std,std-ref}{modindex}

\item {} 
\DUrole{xref,std,std-ref}{search}

\end{itemize}


\renewcommand{\indexname}{Python Module Index}
\begin{sphinxtheindex}
\let\bigletter\sphinxstyleindexlettergroup
\bigletter{f}
\item\relax\sphinxstyleindexentry{fit\_functions}\sphinxstyleindexpageref{fit-functions:\detokenize{module-fit_functions}}
\indexspace
\bigletter{m}
\item\relax\sphinxstyleindexentry{measurement}\sphinxstyleindexpageref{measurement:\detokenize{module-measurement}}
\indexspace
\bigletter{p}
\item\relax\sphinxstyleindexentry{pigor}\sphinxstyleindexpageref{pigor:\detokenize{module-pigor}}
\end{sphinxtheindex}

\renewcommand{\indexname}{Index}
\printindex
\end{document}