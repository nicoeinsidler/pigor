%% Generated by Sphinx.
\def\sphinxdocclass{report}
\documentclass[letterpaper,10pt,english]{sphinxmanual}
\ifdefined\pdfpxdimen
   \let\sphinxpxdimen\pdfpxdimen\else\newdimen\sphinxpxdimen
\fi \sphinxpxdimen=.75bp\relax

\PassOptionsToPackage{warn}{textcomp}
\usepackage[utf8]{inputenc}
\ifdefined\DeclareUnicodeCharacter
% support both utf8 and utf8x syntaxes
\edef\sphinxdqmaybe{\ifdefined\DeclareUnicodeCharacterAsOptional\string"\fi}
  \DeclareUnicodeCharacter{\sphinxdqmaybe00A0}{\nobreakspace}
  \DeclareUnicodeCharacter{\sphinxdqmaybe2500}{\sphinxunichar{2500}}
  \DeclareUnicodeCharacter{\sphinxdqmaybe2502}{\sphinxunichar{2502}}
  \DeclareUnicodeCharacter{\sphinxdqmaybe2514}{\sphinxunichar{2514}}
  \DeclareUnicodeCharacter{\sphinxdqmaybe251C}{\sphinxunichar{251C}}
  \DeclareUnicodeCharacter{\sphinxdqmaybe2572}{\textbackslash}
\fi
\usepackage{cmap}
\usepackage[T1]{fontenc}
\usepackage{amsmath,amssymb,amstext}
\usepackage{babel}
\usepackage{times}
\usepackage[Bjarne]{fncychap}
\usepackage{sphinx}

\fvset{fontsize=\small}
\usepackage{geometry}

% Include hyperref last.
\usepackage{hyperref}
% Fix anchor placement for figures with captions.
\usepackage{hypcap}% it must be loaded after hyperref.
% Set up styles of URL: it should be placed after hyperref.
\urlstyle{same}
\addto\captionsenglish{\renewcommand{\contentsname}{Contents:}}

\addto\captionsenglish{\renewcommand{\figurename}{Fig.\@ }}
\makeatletter
\def\fnum@figure{\figurename\thefigure{}}
\makeatother
\addto\captionsenglish{\renewcommand{\tablename}{Table }}
\makeatletter
\def\fnum@table{\tablename\thetable{}}
\makeatother
\addto\captionsenglish{\renewcommand{\literalblockname}{Listing}}

\addto\captionsenglish{\renewcommand{\literalblockcontinuedname}{continued from previous page}}
\addto\captionsenglish{\renewcommand{\literalblockcontinuesname}{continues on next page}}
\addto\captionsenglish{\renewcommand{\sphinxnonalphabeticalgroupname}{Non-alphabetical}}
\addto\captionsenglish{\renewcommand{\sphinxsymbolsname}{Symbols}}
\addto\captionsenglish{\renewcommand{\sphinxnumbersname}{Numbers}}

\addto\extrasenglish{\def\pageautorefname{page}}

\setcounter{tocdepth}{1}



\title{PIGOR Documentation}
\date{Oct 17, 2019}
\release{}
\author{Nico Einsidler}
\newcommand{\sphinxlogo}{\vbox{}}
\renewcommand{\releasename}{}
\makeindex
\begin{document}

\pagestyle{empty}
\sphinxmaketitle
\pagestyle{plain}
\sphinxtableofcontents
\pagestyle{normal}
\phantomsection\label{\detokenize{index::doc}}


Pigor is a lightweight analysis tool for the the polarimeter instrument NEPTUN beam port of the 250kW research reactor hosted at \sphinxhref{https://ati.tuwien.ac.at/startseite/}{Atominstitut} of \sphinxhref{https://www.tuwien.ac.at/}{TU Wien}, Austria. For more information visit \sphinxhref{http://www.neutroninterferometry.com/}{our homepage}.


\chapter{How to install PIGOR}
\label{\detokenize{installation:how-to-install-pigor}}\label{\detokenize{installation::doc}}

\section{Installing PIGOR}
\label{\detokenize{installation:installing-pigor}}\label{\detokenize{installation:installation}}
Assuming you have a running installation of \sphinxstylestrong{Python 3.7} or higher, to install PIGOR please follow these steps:
\begin{enumerate}
\def\theenumi{\arabic{enumi}}
\def\labelenumi{\theenumi .}
\makeatletter\def\p@enumii{\p@enumi \theenumi .}\makeatother
\item {} \begin{description}
\item[{Download the source from Bitbucket with \sphinxcode{\sphinxupquote{git clone https://github.com/nicoeinsidler/pigor.git}}. This should include (at least) the following files:}] \leavevmode\begin{itemize}
\item {} 
\sphinxcode{\sphinxupquote{measurement.py}}

\item {} 
\sphinxcode{\sphinxupquote{pigor.py}}

\item {} 
\sphinxcode{\sphinxupquote{README.md}}

\item {} 
\sphinxcode{\sphinxupquote{requirements.py}}

\item {} 
\sphinxcode{\sphinxupquote{requirements-dev.py}}

\item {} 
\sphinxcode{\sphinxupquote{requirements-measurement.py}}

\end{itemize}

\end{description}

\item {} 
Install the requirements with:

\begin{sphinxVerbatim}[commandchars=\\\{\}]
\PYG{n}{pip} \PYG{n}{install} \PYG{o}{\PYGZhy{}}\PYG{n}{r} \PYG{n}{requirements}\PYG{o}{.}\PYG{n}{txt}
\end{sphinxVerbatim}

\end{enumerate}

\begin{sphinxadmonition}{note}{Note:}
If you are using two separate python 2.x and 3.x installations, you might need to use \sphinxcode{\sphinxupquote{pip3}} instead of \sphinxcode{\sphinxupquote{pip}}.
\end{sphinxadmonition}

You can now run PIGOR with the following command in the folder where the \sphinxcode{\sphinxupquote{pigor.py}} and \sphinxcode{\sphinxupquote{measurement.py}} files are located:

\begin{sphinxVerbatim}[commandchars=\\\{\}]
\PYG{n}{python} \PYG{n}{pigor}\PYG{o}{.}\PYG{n}{py}
\end{sphinxVerbatim}

\begin{sphinxadmonition}{note}{Note:}
If you only want to use the \sphinxcode{\sphinxupquote{measurement.Measurement}} class, you can only install it’s dependencies with \sphinxcode{\sphinxupquote{pip install -r requirements-measurement.txt}}. But since most dependencies are shared, it won’t make much of a difference.
\end{sphinxadmonition}


\section{Installing only the Measurement Class}
\label{\detokenize{installation:installing-only-the-measurement-class}}
\begin{sphinxadmonition}{warning}{Warning:}
This is not maintained yet. Please use this feature with caution and build it first.
\end{sphinxadmonition}

In some cases the user may only want to install the measurement module to use the Measurement class as one of their python installations modules. A zip or tarball file should be provided for this use case. See \DUrole{xref,std,std-ref}{deployment-measurement-module}.

To install (and copy into the user’s installed modules) simply unpack the zip or tarball. Head over to its directory and type from there:

\begin{sphinxVerbatim}[commandchars=\\\{\}]
python setup.py install
\end{sphinxVerbatim}


\section{Installation for Developers}
\label{\detokenize{installation:installation-for-developers}}
A known working set of python modules used to create PIGOR can be found in \sphinxcode{\sphinxupquote{requirements-dev.txt}}. To install all these requirements, simply type:

\begin{sphinxVerbatim}[commandchars=\\\{\}]
\PYG{n}{pip} \PYG{n}{install} \PYG{o}{\PYGZhy{}}\PYG{n}{r} \PYG{n}{requirements}\PYG{o}{\PYGZhy{}}\PYG{n}{dev}\PYG{o}{.}\PYG{n}{txt}
\end{sphinxVerbatim}

It is recommended to use a virtual environment.


\chapter{Measurement Naming Convention}
\label{\detokenize{naming-convention:measurement-naming-convention}}\label{\detokenize{naming-convention::doc}}
In order for the Measurement class to work properly, the user should follow this naming convention:

\sphinxcode{\sphinxupquote{YYYY\_MM\_DD\_HHmm\_IDENTIFIER\_TYPE.dat}} (not case sensitive)

where…
\begin{itemize}
\item {} \begin{description}
\item[{\sphinxcode{\sphinxupquote{IDENTIFIER}} denotes the type of measurement or the reason for the measurement. The following values are common and recognized by PIGOR:}] \leavevmode\begin{itemize}
\item {} 
\sphinxcode{\sphinxupquote{dcX\#}}: DC coil (for the magnetic field in x-direction) scan for the DC coil with the number \#. A sine fit is automatically applied, when \sphinxcode{\sphinxupquote{TYPE}} is not set.

\item {} 
\sphinxcode{\sphinxupquote{dcZ\#}}: DC coil scan in z-direction (compensation field) for DC coil number \#. Fitted with a polynomial of order 5 by default.

\item {} 
\sphinxcode{\sphinxupquote{pos\#}}: position scan with stage number \#. Fitted with a Gaussian by default.

\end{itemize}

\end{description}

\item {} \begin{description}
\item[{\sphinxcode{\sphinxupquote{TYPE}} specifies and overrides the type of fitting/analysis that should be applied by PIGOR. Possible values:}] \leavevmode\begin{itemize}
\item {} \begin{description}
\item[{fitting types}] \leavevmode\begin{itemize}
\item {} 
\sphinxcode{\sphinxupquote{poly5}}: polynom of order 5

\item {} 
\sphinxcode{\sphinxupquote{sine\_lin}}: sine plus additional added linear function

\item {} 
\sphinxcode{\sphinxupquote{sine}}: sine

\item {} 
\sphinxcode{\sphinxupquote{gauss}}: Gauß fit

\end{itemize}

\end{description}

\item {} \begin{description}
\item[{special analysis types}] \leavevmode\begin{itemize}
\item {} 
\sphinxcode{\sphinxupquote{pol}}: specifies that this measurement is a degree of polarisation measurement. This will automatically calculate the degree of polarisation for any given number of points (number of data points must be multiple of four)

\end{itemize}

\end{description}

\end{itemize}

\end{description}

\end{itemize}

Types can be combined within a types group (gauss and pol for example, like “gauss\_pol” or “pol\_gauss”).


\chapter{Quickstart Guide}
\label{\detokenize{quickstart:quickstart-guide}}\label{\detokenize{quickstart::doc}}
\begin{sphinxadmonition}{note}{Note:}
Please refer to the {\hyperref[\detokenize{installation:installation}]{\sphinxcrossref{\DUrole{std,std-ref}{Installing PIGOR}}}} page for more information on the installation.
\end{sphinxadmonition}

Suppose you have some measurement data files with in a directory. The hierachy looks like this:
\begin{itemize}
\item {} \begin{description}
\item[{measurements}] \leavevmode\begin{itemize}
\item {} \begin{description}
\item[{2019-01}] \leavevmode\begin{itemize}
\item {} 
2019-01-01-dc1x.dat

\item {} 
2019-01-04-dc1z.dat

\item {} 
2019-01-05-dc1z.dat

\end{itemize}

\end{description}

\item {} \begin{description}
\item[{2019-02}] \leavevmode\begin{itemize}
\item {} 
2019-02-10-dc2x.dat

\item {} 
2019-02-11-dc2x.dat

\item {} 
2019-02-14-dc2z.dat

\item {} 
2019-01-19-dc2z.dat

\item {} 
2019-01-20-dc2z.dat

\item {} 
2019-01-28-dc1x.dat

\end{itemize}

\end{description}

\item {} \begin{description}
\item[{2019-03}] \leavevmode\begin{itemize}
\item {} 
2019-02-02-dc3x.dat

\item {} 
2019-02-03-dc3z.dat

\end{itemize}

\end{description}

\end{itemize}

\end{description}

\end{itemize}

\begin{sphinxadmonition}{note}{Note:}
Here we will assume that \sphinxcode{\sphinxupquote{python}} will start Python 3. On some installations it must be explicitly started with \sphinxcode{\sphinxupquote{python3}}.
\end{sphinxadmonition}

In order to analyse these files with PIGOR, head to the directory where PIGOR is located. Start it by executing the command \sphinxcode{\sphinxupquote{python pigor.py}}. If it is the first startup it will go through some questions regarding your analysis setup. The most important question is the first one: Where should PIGOR look for files to analyse? This is the folder where all your measurements are located in and will be refered as PIGOR’s root folder. In our example type in the path to \sphinxcode{\sphinxupquote{measurements/}}. You can either use absolute paths or relative ones. PIGOR will automatically check if the path actually exists, if not you will be prompted to enter a new one again. It also does not matter if you are using a POSIX type (IEEE Std 1003.1-1988) or as a DOS type path. (POSIX uses / and DOS uses as separators)

If PIGOR does not ask these question on startup, they have already been configured. Either delete the \sphinxcode{\sphinxupquote{pigor.config}} file in the directory where \sphinxcode{\sphinxupquote{pigor.py}} is located, or just use the command {[}i{]} within PIGOR to reconfigure it.

Now you should see something like that, depending on PIGOR’s version:

\begin{sphinxVerbatim}[commandchars=\\\{\}]
\PYGZdl{} python pigor.py

\PYG{o}{=}\PYG{o}{=}\PYG{o}{=}\PYG{o}{=}\PYG{o}{=}\PYG{o}{=}\PYG{o}{=}\PYG{o}{=}\PYG{o}{=}\PYG{o}{=}\PYG{o}{=}\PYG{o}{=}\PYG{o}{=}\PYG{o}{=}\PYG{o}{=}\PYG{o}{=}\PYG{o}{=}\PYG{o}{=}\PYG{o}{=}\PYG{o}{=}\PYG{o}{=}\PYG{o}{=}
Welcome to PIGOR v1.1.
\PYG{o}{=}\PYG{o}{=}\PYG{o}{=}\PYG{o}{=}\PYG{o}{=}\PYG{o}{=}\PYG{o}{=}\PYG{o}{=}\PYG{o}{=}\PYG{o}{=}\PYG{o}{=}\PYG{o}{=}\PYG{o}{=}\PYG{o}{=}\PYG{o}{=}\PYG{o}{=}\PYG{o}{=}\PYG{o}{=}\PYG{o}{=}\PYG{o}{=}\PYG{o}{=}\PYG{o}{=}

i ... init
h ... print\PYGZus{}help
a ... analyse\PYGZus{}files
j ... create\PYGZus{}index
r ... remove\PYGZus{}generated\PYGZus{}files
x ... print\PYGZus{}root
q ... quit PIGOR v1.1

PIGOR v1.1 will look \PYG{k}{for} measurement files in D:\PYG{l+s+se}{\PYGZbs{}m}easurements.

If you need more information about a command, just \PYG{n+nb}{type} h + \PYG{o}{[}command\PYG{o}{]} + \PYGZlt{}ENTER\PYGZgt{}
to get more help. For example: h + a + \PYGZlt{}ENTER\PYGZgt{}.

Please \PYG{n+nb}{type} a \PYG{n+nb}{command} you want to perform and press \PYGZlt{}ENTER\PYGZgt{}.
\end{sphinxVerbatim}

On startup PIGOR is printing the root, which in this case is \sphinxcode{\sphinxupquote{D:\textbackslash{}measurements}}. It will also include a list of available commands. Each command can be triggered by pressing the corresponding letter followed by an \textless{}ENTER\textgreater{}. Some commands allow more options, each separated by a space.

We will describe these commands by {[}cmd{]} where cmd stands for the specific command.

To get more help on a specific command, just type {[}h{]} + {[}cmd{]}. For example typing \sphinxcode{\sphinxupquote{h a}} will give us the help on \sphinxcode{\sphinxupquote{analyse\_files()}}. Let’s try it:

\begin{sphinxVerbatim}[commandchars=\\\{\}]
Please \PYG{n+nb}{type} a \PYG{n+nb}{command} you want to perform and press \PYGZlt{}ENTER\PYGZgt{}.
h a
a:

Analyses all given files in list. This \PYG{k}{function} can be used by the \PYG{n+nb}{command} \PYG{o}{[}a\PYG{o}{]}.

    :param filepaths:   list of files to analyse with
                        their relative dir path added

    .. todo:: Change to no override mode. measurement.Measurement.plot\PYG{o}{(}\PYG{n+nv}{override}\PYG{o}{=}False\PYG{o}{)}
    .. todo:: a + \PYG{n+nv}{today} \PYG{o}{=}\PYGZgt{} only analyse files \PYG{k}{for} today
    .. todo:: a + \PYG{n+nv}{override} \PYG{o}{=}\PYGZgt{} \PYG{n+nv}{override}\PYG{o}{=}True
\end{sphinxVerbatim}

This is the same help text as found in this documentation.

We can now analyse our files with {[}a{]}. You can see on which file PIGOR is currently working on. If an error occurs, you will see it as well and PIGOR will skip the file.

After PIGOR is done analysing files, you may want to access these files. Use the {[}j{]} command to create an index to quickly go through all the files that have been analysed. Now you should see a new file has been created:
\begin{itemize}
\item {} \begin{description}
\item[{measurements}] \leavevmode\begin{itemize}
\item {} 
index\_pigor.html

\item {} 
2019-01

\item {} 
2019-02

\item {} 
2019-03

\end{itemize}

\end{description}

\end{itemize}

Open \sphinxcode{\sphinxupquote{index\_pigor.html}} to see the list. From there you can review the original data and the files that PIGOR created from this.


\chapter{PIGOR}
\label{\detokenize{pigor:pigor}}\label{\detokenize{pigor::doc}}

\section{What PIGOR does}
\label{\detokenize{pigor:what-pigor-does}}
PIGOR (‘Python IGOR’ = PIGOR) aims to help physicists on the NEPTUN beamline to quickly extract the needed information when measuring and configuring or preparing an experiment. It will go through all files in its root folder and will continue to \sphinxstylestrong{look for files in all subdirectories recursively} as well. It will then \sphinxstylestrong{auto detect} %
\begin{footnote}[1]\sphinxAtStartFootnote
This will only work if the correct naming convention is used.
%
\end{footnote} \sphinxstylestrong{the type of measurement} and guess what the user wants to know. After analysis of all files, \sphinxstylestrong{additional files will show up alongside the original measurement files}:
\begin{itemize}
\item {} 
.png file: plot of the data

\item {} 
.md file: usefull information gathered about the measurement in plain text as markdown

\item {} 
.html file: same content as the markdown file, but nicely viewable in a modern browser

\end{itemize}


\section{PIGORs inner workings}
\label{\detokenize{pigor:pigors-inner-workings}}

\chapter{The Measurement Class}
\label{\detokenize{measurement:the-measurement-class}}\label{\detokenize{measurement::doc}}

\section{Flow at Startup}
\label{\detokenize{measurement:flow-at-startup}}\begin{enumerate}
\def\theenumi{\arabic{enumi}}
\def\labelenumi{\theenumi .}
\makeatletter\def\p@enumii{\p@enumi \theenumi .}\makeatother
\item {} 
read the data

\item {} 
detect measurement type

\item {} 
read position file if a position file exists

\item {} 
clean up and description gathering

\item {} 
select columns for plotting

\end{enumerate}


\section{Flow when plotting}
\label{\detokenize{measurement:flow-when-plotting}}

\section{Class Usecases}
\label{\detokenize{measurement:class-usecases}}
There are many ways to interact with or use the Measurement class. Here are the three main ways:


\section{Methods}
\label{\detokenize{measurement:methods}}
\begin{sphinxadmonition}{note}{\label{measurement:index-0}Todo:}
Method attributes are shown, but value is always None.
\end{sphinxadmonition}


\chapter{Measurement Class 2.0}
\label{\detokenize{measurement2:measurement-class-2-0}}\label{\detokenize{measurement2::doc}}
Due to some design flaws in the existsing \sphinxcode{\sphinxupquote{measurement.Measurement}} class, it will be rewritten from ground up using already existsing libraries. This undertaking will lead to a cleaner version and will make it possible to adapt \sphinxcode{\sphinxupquote{measurement.Measurement}} more easily for the interferometer experiments.


\section{Structure}
\label{\detokenize{measurement2:structure}}
The new immproved structure will make use of:
\begin{itemize}
\item {} 
LMFIT package

\item {} 
pandas dataframe

\end{itemize}

So that the fitting will be done with LMFIT models, whereas the data is handled in pandas dataframes. This creates huge advantages for the developer as well as for the user.


\section{Previous Development}
\label{\detokenize{measurement2:previous-development}}
There have been efforts previously to build a unique own modular \sphinxcode{\sphinxupquote{measurement.Measurement}} class by Nico. These efforts can be examined in the branch improvements/core.

These following elements were attemted to build:
\begin{itemize}
\item {} 
data column: has data and a head, knows its name etc.; functions can easily be applied to it

\item {} 
fit object: used for fitting and finding bounds; each instance can have its own bounds

\item {} \begin{description}
\item[{data set: these objects can be plotted by Measurement, so Measurement will try to create one of those objects; they consists of:}] \leavevmode\begin{itemize}
\item {} 
data columns objects

\item {} 
fit objects

\end{itemize}

\end{description}

\end{itemize}

\sphinxstylestrong{Column Class}

Methods:
\begin{itemize}
\item {} 
\sphinxcode{\sphinxupquote{reverse()}}: reverse order of data

\item {} 
\sphinxcode{\sphinxupquote{\_\_init\_\_(self, desc, data)}}

\item {} 
\sphinxcode{\sphinxupquote{\_\_repr\_\_()}}: plots \sphinxcode{\sphinxupquote{'\textless{}column object 'desc' of lenght len(data)\textgreater{}'}} or something like that

\end{itemize}

Variables:
\begin{itemize}
\item {} 
\sphinxcode{\sphinxupquote{columns.data}}: holds the data as numpy array in float64

\item {} 
\sphinxcode{\sphinxupquote{columns.desc}}: holds the name of the columns heading as string

\end{itemize}

\sphinxstylestrong{Fit Function Class}

Method:
\begin{itemize}
\item {} 
\sphinxcode{\sphinxupquote{fit\_function()}}

\item {} 
\sphinxcode{\sphinxupquote{find\_bounds()}}: tries to find the bounds

\item {} 
\sphinxcode{\sphinxupquote{bounds}}: holds the bounds to be used when fitting as array of tuples

\end{itemize}

Variables:
\begin{itemize}
\item {} 
\sphinxcode{\sphinxupquote{parameters}}: dictionary holding the names of the parameters and the parameters themselves

\end{itemize}

\sphinxstylestrong{Fit Class}

Variables:
\begin{itemize}
\item {} 
\sphinxcode{\sphinxupquote{type}} with which function the fit should be carried out, string

\item {} 
\sphinxcode{\sphinxupquote{popt}}

\item {} 
\sphinxcode{\sphinxupquote{pcov}}

\end{itemize}

Methods:
\begin{itemize}
\item {} 
\sphinxcode{\sphinxupquote{fit()}}

\end{itemize}

\sphinxstylestrong{Data Set Class}

This object stores data points (lists or Column objects) to form a data set. It must contain at least two data point lists. These lists must have the same number of elements, if not the lists that don’t have enough elements will get padded with 0.

Variables:
\begin{itemize}
\item {} 
\sphinxcode{\sphinxupquote{desc}}: a description of what the data set describes (optional)

\item {} 
\sphinxcode{\sphinxupquote{data}}: data is stored in list; len(list) \textgreater{} 1;

\end{itemize}

It is obvious to every reader that indeed almost all of this functionality is already included in the python package pandas and lmfit. This led to the conclusion that improvements/core won’t be maintained any longer.


\section{The brand new Measurement Class}
\label{\detokenize{measurement2:the-brand-new-measurement-class}}
The brand new measurement class will include only the following featues:
\begin{itemize}
\item {} 
ready-to-use lmfit fit models

\item {} 
fit() method to actually produce a fit using lmfits minimize()

\item {} 
plot() method for implementing plotting functionality

\item {} 
contrast() to calculate the contrast

\item {} 
degree\_of\_polarisation() to calculate the degree of polarisation

\item {} 
read() to read data in a very generic way

\item {} 
export\_meta() to export meta data

\end{itemize}

Its child classes will then implement the experiment specific data handling details like cleaning the data and extending the meta data that will be exported.


\chapter{Fit Functions}
\label{\detokenize{fit-functions:fit-functions}}\label{\detokenize{fit-functions::doc}}
The following fit functions are implemented and can be used within PIGOR or the Measurement class. They can be found in \sphinxcode{\sphinxupquote{fit\_functions.py}}.

\phantomsection\label{\detokenize{fit-functions:module-fit_functions}}\index{fit\_functions (module)@\spxentry{fit\_functions}\spxextra{module}}\index{gauss() (in module fit\_functions)@\spxentry{gauss()}\spxextra{in module fit\_functions}}

\begin{fulllineitems}
\phantomsection\label{\detokenize{fit-functions:fit_functions.gauss}}\pysiglinewithargsret{\sphinxcode{\sphinxupquote{fit\_functions.}}\sphinxbfcode{\sphinxupquote{gauss}}}{\emph{x}, \emph{a}, \emph{x0}, \emph{sigma}, \emph{export=False}}{}
Gaussian function, used for fitting data.
\begin{quote}\begin{description}
\item[{Parameters}] \leavevmode\begin{itemize}
\item {} 
\sphinxstyleliteralstrong{\sphinxupquote{x}} \textendash{} parameter

\item {} 
\sphinxstyleliteralstrong{\sphinxupquote{a}} \textendash{} amplitude

\item {} 
\sphinxstyleliteralstrong{\sphinxupquote{x0}} \textendash{} maximum

\item {} 
\sphinxstyleliteralstrong{\sphinxupquote{sigma}} \textendash{} width

\item {} 
\sphinxstyleliteralstrong{\sphinxupquote{export}} \textendash{} enable text output of function

\end{itemize}

\end{description}\end{quote}

\end{fulllineitems}

\index{poly() (in module fit\_functions)@\spxentry{poly()}\spxextra{in module fit\_functions}}

\begin{fulllineitems}
\phantomsection\label{\detokenize{fit-functions:fit_functions.poly}}\pysiglinewithargsret{\sphinxcode{\sphinxupquote{fit\_functions.}}\sphinxbfcode{\sphinxupquote{poly}}}{\emph{x}, \emph{*args}, \emph{export=False}}{}
Polynom nth degree for fitting.
\begin{quote}\begin{description}
\item[{Parameters}] \leavevmode\begin{itemize}
\item {} 
\sphinxstyleliteralstrong{\sphinxupquote{x}} (\sphinxstyleliteralemphasis{\sphinxupquote{int}}\sphinxstyleliteralemphasis{\sphinxupquote{, }}\sphinxstyleliteralemphasis{\sphinxupquote{float}}) \textendash{} parameter

\item {} 
\sphinxstyleliteralstrong{\sphinxupquote{*args}} \textendash{} 
list of coefficients {[}a\_N,a\_N-1, …, a\_1, a\_0{]}


\item {} 
\sphinxstyleliteralstrong{\sphinxupquote{export}} (\sphinxstyleliteralemphasis{\sphinxupquote{bool}}\sphinxstyleliteralemphasis{\sphinxupquote{ or }}\sphinxstyleliteralemphasis{\sphinxupquote{string}}\sphinxstyleliteralemphasis{\sphinxupquote{, }}\sphinxstyleliteralemphasis{\sphinxupquote{optional}}) \textendash{} enable text output of function, defaults to False

\end{itemize}

\item[{Returns}] \leavevmode
returns the polynomial

\item[{Return type}] \leavevmode
str, int, float

\end{description}\end{quote}

\begin{sphinxVerbatim}[commandchars=\\\{\}]
\PYG{g+gp}{\PYGZgt{}\PYGZgt{}\PYGZgt{} }\PYG{n}{poly}\PYG{p}{(}\PYG{l+m+mf}{3.4543}\PYG{p}{,} \PYG{l+m+mi}{5}\PYG{p}{,}\PYG{l+m+mi}{4}\PYG{p}{,}\PYG{l+m+mi}{3}\PYG{p}{,}\PYG{l+m+mi}{2}\PYG{p}{,}\PYG{l+m+mi}{1}\PYG{p}{,} \PYG{n}{export}\PYG{o}{=}\PYG{l+s+s1}{\PYGZsq{}}\PYG{l+s+s1}{Mathematica}\PYG{l+s+s1}{\PYGZsq{}}\PYG{p}{)}
\PYG{g+go}{\PYGZsq{}5*3.4543\PYGZca{}5 + 4*3.4543\PYGZca{}4 + 3*3.4543\PYGZca{}3 + 2*3.4543\PYGZca{}2 + 1*3.4543\PYGZca{}1\PYGZsq{}}
\end{sphinxVerbatim}

\begin{sphinxVerbatim}[commandchars=\\\{\}]
\PYG{g+gp}{\PYGZgt{}\PYGZgt{}\PYGZgt{} }\PYG{n}{poly}\PYG{p}{(}\PYG{l+m+mf}{3.4543}\PYG{p}{,} \PYG{l+m+mi}{5}\PYG{p}{,}\PYG{l+m+mi}{4}\PYG{p}{,}\PYG{l+m+mi}{3}\PYG{p}{,}\PYG{l+m+mi}{2}\PYG{p}{,}\PYG{l+m+mi}{1}\PYG{p}{)}
\PYG{g+go}{920.4602110784704}
\end{sphinxVerbatim}

\end{fulllineitems}

\index{poly5() (in module fit\_functions)@\spxentry{poly5()}\spxextra{in module fit\_functions}}

\begin{fulllineitems}
\phantomsection\label{\detokenize{fit-functions:fit_functions.poly5}}\pysiglinewithargsret{\sphinxcode{\sphinxupquote{fit\_functions.}}\sphinxbfcode{\sphinxupquote{poly5}}}{\emph{x}, \emph{a5}, \emph{a4}, \emph{a3}, \emph{a2}, \emph{a1}, \emph{a0}, \emph{export=False}}{}
Polynom 5th degree for fitting.
\begin{quote}\begin{description}
\item[{Parameters}] \leavevmode\begin{itemize}
\item {} 
\sphinxstyleliteralstrong{\sphinxupquote{x}} \textendash{} parameter

\item {} 
\sphinxstyleliteralstrong{\sphinxupquote{a5}} \textendash{} coeff

\item {} 
\sphinxstyleliteralstrong{\sphinxupquote{a4}} \textendash{} coeff

\item {} 
\sphinxstyleliteralstrong{\sphinxupquote{a3}} \textendash{} coeff

\item {} 
\sphinxstyleliteralstrong{\sphinxupquote{a2}} \textendash{} coeff

\item {} 
\sphinxstyleliteralstrong{\sphinxupquote{a1}} \textendash{} coeff

\item {} 
\sphinxstyleliteralstrong{\sphinxupquote{a0}} \textendash{} coeff

\item {} 
\sphinxstyleliteralstrong{\sphinxupquote{export}} \textendash{} enable text output of function

\end{itemize}

\item[{Returns}] \leavevmode
function \textendash{} polynomial 5th degree

\end{description}\end{quote}

\end{fulllineitems}

\index{register\_fit\_function() (in module fit\_functions)@\spxentry{register\_fit\_function()}\spxextra{in module fit\_functions}}

\begin{fulllineitems}
\phantomsection\label{\detokenize{fit-functions:fit_functions.register_fit_function}}\pysiglinewithargsret{\sphinxcode{\sphinxupquote{fit\_functions.}}\sphinxbfcode{\sphinxupquote{register\_fit\_function}}}{\emph{func}, \emph{bounds=(-inf}, \emph{inf)}}{}
This decorator registers a new fit function and writes an entry to fit\_function\_list.

\end{fulllineitems}

\index{sine() (in module fit\_functions)@\spxentry{sine()}\spxextra{in module fit\_functions}}

\begin{fulllineitems}
\phantomsection\label{\detokenize{fit-functions:fit_functions.sine}}\pysiglinewithargsret{\sphinxcode{\sphinxupquote{fit\_functions.}}\sphinxbfcode{\sphinxupquote{sine}}}{\emph{x}, \emph{a}, \emph{omega}, \emph{phase}, \emph{c}, \emph{export=False}}{}
Sine function for fitting data.
\begin{quote}\begin{description}
\item[{Parameters}] \leavevmode\begin{itemize}
\item {} 
\sphinxstyleliteralstrong{\sphinxupquote{x}} \textendash{} parameter

\item {} 
\sphinxstyleliteralstrong{\sphinxupquote{a}} \textendash{} amplitude

\item {} 
\sphinxstyleliteralstrong{\sphinxupquote{omega}} \textendash{} frequency

\item {} 
\sphinxstyleliteralstrong{\sphinxupquote{phase}} \textendash{} phase

\item {} 
\sphinxstyleliteralstrong{\sphinxupquote{c}} \textendash{} offset

\item {} 
\sphinxstyleliteralstrong{\sphinxupquote{export}} \textendash{} enable text output of function

\end{itemize}

\end{description}\end{quote}

\end{fulllineitems}

\index{sine\_lin() (in module fit\_functions)@\spxentry{sine\_lin()}\spxextra{in module fit\_functions}}

\begin{fulllineitems}
\phantomsection\label{\detokenize{fit-functions:fit_functions.sine_lin}}\pysiglinewithargsret{\sphinxcode{\sphinxupquote{fit\_functions.}}\sphinxbfcode{\sphinxupquote{sine\_lin}}}{\emph{x}, \emph{a}, \emph{omega}, \emph{phase}, \emph{c}, \emph{b}, \emph{export=False}}{}
Sine function with linear term added for fitting data.
\begin{quote}\begin{description}
\item[{Parameters}] \leavevmode\begin{itemize}
\item {} 
\sphinxstyleliteralstrong{\sphinxupquote{x}} \textendash{} parameter

\item {} 
\sphinxstyleliteralstrong{\sphinxupquote{a}} \textendash{} amplitude

\item {} 
\sphinxstyleliteralstrong{\sphinxupquote{omega}} \textendash{} frequency

\item {} 
\sphinxstyleliteralstrong{\sphinxupquote{phase}} \textendash{} phase

\item {} 
\sphinxstyleliteralstrong{\sphinxupquote{c}} \textendash{} offset

\item {} 
\sphinxstyleliteralstrong{\sphinxupquote{b}} \textendash{} slope

\item {} 
\sphinxstyleliteralstrong{\sphinxupquote{export}} \textendash{} enable text output of function

\end{itemize}

\end{description}\end{quote}

\end{fulllineitems}



\chapter{Sprint Planning}
\label{\detokenize{sprints:sprint-planning}}\label{\detokenize{sprints::doc}}
This site gives a quick overview what will come next. Each sprint should take about 1 week to finish.


\section{PIGOR}
\label{\detokenize{sprints:pigor}}\begin{enumerate}
\def\theenumi{\arabic{enumi}}
\def\labelenumi{\theenumi .}
\makeatletter\def\p@enumii{\p@enumi \theenumi .}\makeatother
\item {} \begin{description}
\item[{Sprint}] \leavevmode\begin{itemize}
\item {} 
\(\checkmark\) feature: remove all generated files (html, md, png)

\item {} 
\(\checkmark\) feature: introducing a config file (PIGOR start directory, …)

\item {} 
\(\checkmark\) improvement: auto create config file if not present

\item {} 
\(\checkmark\) improvement: auto register all functions for help menu (decorators)

\end{itemize}

\end{description}

\item {} \begin{description}
\item[{Sprint}] \leavevmode\begin{itemize}
\item {} 
feature: remove last generated files (html, md, png)

\item {} 
feature: remove all html/md or png files

\item {} 
\(\checkmark\) improvement: use JSON for config file

\end{itemize}

\end{description}

\item {} \begin{description}
\item[{Sprint:}] \leavevmode\begin{itemize}
\item {} 
feature: auto run command in specified intervals; syntax maybe: time + {[}cmd{]} + \textless{}ENTER\textgreater{}

\end{itemize}

\end{description}

\end{enumerate}


\section{Measurement Class}
\label{\detokenize{sprints:measurement-class}}\begin{enumerate}
\def\theenumi{\arabic{enumi}}
\def\labelenumi{\theenumi .}
\makeatletter\def\p@enumii{\p@enumi \theenumi .}\makeatother
\item {} \begin{description}
\item[{Sprint}] \leavevmode\begin{itemize}
\item {} 
\(\checkmark\) improvement: switching from self.y \textendash{}\textgreater{} self.y{[}{]} and self.y\_error \textendash{}\textgreater{} self.y\_error{[}{]}

\item {} 
\(\checkmark\) {[}not tested yet{]} improvement: plot multiple self.y’s

\item {} 
feature: auto detect interferometer measurements

\end{itemize}

\end{description}

\item {} \begin{description}
\item[{Sprint}] \leavevmode\begin{itemize}
\item {} 
feature: remove all associated files from file system, except the measurement file itself

\item {} 
\(\checkmark\) improvement: auto register all available fit functions via decorators

\item {} 
improvement: adding \_\_repr\_\_

\end{itemize}

\end{description}

\item {} \begin{description}
\item[{Sprint: finish branch \sphinxcode{\sphinxupquote{feature/interferometer}}}] \leavevmode\begin{itemize}
\item {} 
fixing / understanding inheritance of instance variables (see python test file in branch)

\item {} \begin{description}
\item[{creating subclasses from Measurement:}] \leavevmode\begin{itemize}
\item {} 
Interferometer: adding custom maps to COLUMN\_MAPS and overriding clean\_data() and detect\_measurement()

\item {} 
Polarimeter: adding custom maps to COLUMN\_MAPS and overriding clean\_data() and detect\_measurement()

\end{itemize}

\end{description}

\end{itemize}

\end{description}

\item {} 
Sprint: not planned yet

\end{enumerate}


\section{Ideas}
\label{\detokenize{sprints:ideas}}
Building Measurement from ground up with custom objects like:
\begin{itemize}
\item {} 
data column: has data and a head, knows its name etc.; functions can easily be applied to it

\item {} 
fit object: used for fitting and finding bounds; each instance can have its own bounds

\item {} \begin{description}
\item[{data set: these objects can be plotted by Measurement, so Measurement will try to create one of those objects; they consists of:}] \leavevmode\begin{itemize}
\item {} 
data columns objects

\item {} 
fit objects

\end{itemize}

\end{description}

\end{itemize}


\section{Column Class}
\label{\detokenize{sprints:column-class}}
Methods:
\begin{itemize}
\item {} 
\sphinxcode{\sphinxupquote{reverse()}}: reverse order of data

\item {} 
\sphinxcode{\sphinxupquote{\_\_init\_\_(self, desc, data)}}

\item {} 
\sphinxcode{\sphinxupquote{\_\_repr\_\_()}}: plots \sphinxcode{\sphinxupquote{'\textless{}column object 'desc' of lenght len(data)\textgreater{}'}} or something like that

\end{itemize}

Variables:
\begin{itemize}
\item {} 
\sphinxcode{\sphinxupquote{columns.data}}: holds the data as numpy array in float64

\item {} 
\sphinxcode{\sphinxupquote{columns.desc}}: holds the name of the columns heading as string

\end{itemize}


\section{Fit Function Class}
\label{\detokenize{sprints:fit-function-class}}
Method:
\begin{itemize}
\item {} 
\sphinxcode{\sphinxupquote{fit\_function()}}

\item {} 
\sphinxcode{\sphinxupquote{find\_bounds()}}: tries to find the bounds

\item {} 
\sphinxcode{\sphinxupquote{bounds}}: holds the bounds to be used when fitting as array of tuples

\end{itemize}

Variables:
\begin{itemize}
\item {} 
\sphinxcode{\sphinxupquote{parameters}}: dictionary holding the names of the parameters and the parameters themselves

\end{itemize}


\section{Fit Class}
\label{\detokenize{sprints:fit-class}}
Variables:
\begin{itemize}
\item {} 
\sphinxcode{\sphinxupquote{type}} with which function the fit should be carried out, string

\item {} 
\sphinxcode{\sphinxupquote{popt}}

\item {} 
\sphinxcode{\sphinxupquote{pcov}}

\end{itemize}

Methods:
\begin{itemize}
\item {} 
\sphinxcode{\sphinxupquote{fit()}}

\end{itemize}


\section{Data Set Class}
\label{\detokenize{sprints:data-set-class}}
This object stores data points (lists or Column objects) to form a data set. It must contain at least two data point lists. These lists must have the same number of elements, if not the lists that don’t have enough elements will get padded with 0.

Variables:
\begin{itemize}
\item {} 
\sphinxcode{\sphinxupquote{desc}}: a description of what the data set describes (optional)

\item {} 
\sphinxcode{\sphinxupquote{data}}: data is stored in list; len(list) \textgreater{} 1;

\end{itemize}


\chapter{ToDo List}
\label{\detokenize{todo:todo-list}}\label{\detokenize{todo::doc}}
\begin{sphinxadmonition}{note}{Todo:}
check if dependencies are correct with dependencies.txt
\end{sphinxadmonition}

(The {\hyperref[\detokenize{index:index-0}]{\sphinxcrossref{\sphinxstyleemphasis{original entry}}}} is located in /Users/nicoeinsidler/Code/pigor/doc/index.rst, line 55.)

\begin{sphinxadmonition}{note}{Todo:}
Method attributes are shown, but value is always None.
\end{sphinxadmonition}

(The {\hyperref[\detokenize{measurement:index-0}]{\sphinxcrossref{\sphinxstyleemphasis{original entry}}}} is located in /Users/nicoeinsidler/Code/pigor/doc/measurement.rst, line 62.)


\chapter{Old ToDos}
\label{\detokenize{index:old-todos}}
\begin{sphinxadmonition}{note}{Note:}
This list of ToDos should be integrated into the docstrings or into the sprint planning where those ToDos belong.
\end{sphinxadmonition}

Here are all ToDos listed. Feel free to contribute and check this project out on Bitbucket.
\begin{itemize}
\item {} 
self.x\_error: not yet implemented

\item {} 
a lot of commenting

\item {} 
error of fit

\item {} 
verbose mode on/off

\item {} 
getting PIGOR ready for shipment by creating a setup.py

\item {} 
better display of self.pcov

\item {} 
separation between pure functions and functions with context (methods)

\item {} 
auto comment decorator for functions

\item {} 
use decorators to auto register fit functions with their input argument list

\item {} 
setuptools in setup.py

\item {} 
\sphinxurl{https://click.palletsprojects.com/en/7.x/quickstart/}

\end{itemize}


\chapter{Project Dependencies}
\label{\detokenize{index:project-dependencies}}
\begin{sphinxadmonition}{note}{\label{index:index-0}Todo:}
check if dependencies are correct with dependencies.txt
\end{sphinxadmonition}
\begin{itemize}
\item {} 
numpy

\item {} 
re

\item {} 
matplotlib

\item {} 
os.path

\item {} 
glob

\item {} 
difflib

\item {} 
datetime

\item {} 
pathlib

\item {} 
scipy

\item {} 
markdown

\end{itemize}


\chapter{Indices and tables}
\label{\detokenize{index:indices-and-tables}}\begin{itemize}
\item {} 
\DUrole{xref,std,std-ref}{genindex}

\item {} 
\DUrole{xref,std,std-ref}{modindex}

\item {} 
\DUrole{xref,std,std-ref}{search}

\end{itemize}


\renewcommand{\indexname}{Python Module Index}
\begin{sphinxtheindex}
\let\bigletter\sphinxstyleindexlettergroup
\bigletter{f}
\item\relax\sphinxstyleindexentry{fit\_functions}\sphinxstyleindexpageref{fit-functions:\detokenize{module-fit_functions}}
\end{sphinxtheindex}

\renewcommand{\indexname}{Index}
\printindex
\end{document}